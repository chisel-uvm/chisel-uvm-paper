% DFF template for Latex and A4 paper.
% 12pt Times New Roman on 1.5 line spacing and 2 cm margins.

% ----------------------------------------------------------------------

% Either format with
%    pdflatex projectdescription.tex
% Or if you use dvips and ps2pdf, remember to specify A4 paper:
%    latex  projectdescription
%    dvips  -ta4 projectdescription -o projectdescription.ps
%    ps2pdf -sPAPERSIZE=a4 projectdescription.ps

% ----------------------------------------------------------------------

\documentclass[fleqn,12pt]{article}
\usepackage[a4paper,top=2cm,bottom=2cm,left=2cm,right=2cm]{geometry}
\usepackage{times}
\usepackage[danish,english]{babel}
\usepackage[utf8]{inputenc}
\usepackage[T1]{fontenc}
% \usepackage{graphicx}         % For PDF figures
% \usepackage[dvips]{graphicx}  % For EPS figures, using dvips + ps2pdf

\usepackage[colorlinks=true,linkcolor=black,citecolor=black]{hyperref}
\usepackage{booktabs}

\usepackage{tikz}
\usetikzlibrary{positioning,fit}
\usetikzlibrary{shapes,backgrounds}
\usetikzlibrary{arrows,fit,automata,positioning,decorations,calc}
\usetikzlibrary{spy}
\usetikzlibrary{matrix,chains,decorations.pathreplacing}
\usepackage{pgfgantt}

\newcommand{\code}[1]{{\textsf{#1}}}

% Adding comments in the text during writing process
\newcommand{\todo}[1]{{\it TODO: #1}}
\newcommand{\note}[1]{{\it Note: #1}}
\newcommand{\martin}[1]{{\color{blue} Martin: #1}}
\newcommand{\jens}[1]{{\color{green} Jens: #1}}

% uncomment following for final submission
%\renewcommand{\todo}[1]{}
%\renewcommand{\note}[1]{}
%\renewcommand{\martin}[1]{}
%\renewcommand{\jens}[1]{}

\usepackage{acronym}

\acrodef{MLP}{Multi-Layer Perceptron}
\acrodef{ANN}{Artificial Neural Network}
\acrodef{CNN}{Convolutional Neural Network}
\acrodef{AI}{Artificial Intelligence}
\acrodef{NN}{Neural Network}
\acrodef{RNN}{Recurrent Neural Network}
\acrodef{SMT}{Satisfiability Modulo Theory}
\acrodef{DNN}{Deep Neural Network}
\acrodef{CPS}{Cyber-Physical System}
\acrodef{SANN}{Synchronous Artificial Neural Network}
\acrodef{WCRT}{Worst Case Reaction Time}

\begin{document}
% Empirically this seems to match MS Word's idea of 1.5 line spacing.
% DO NOT CHANGE
\setlength{\baselineskip}{1.44\baselineskip}

% ----------------------------------------------------------------------
% Enter the title of the project and your name

\begin{center}
  {\LARGE\bf DFF Project Description }\\[1ex]
  {\LARGE\bf  Chisel or High-Level Verification of Digital Systems\\
  or Software Defined Hardware}\\[1ex]
  {\large Martin Schoeberl, DTU Compute}\\[1ex]
 \end{center}

% ----------------------------------------------------------------------
% Delete the instruction

%\noindent
%The length of the project description must not exceed the number of pages indicated for the specific instrument in the Call, excl. a brief list of references, whether it includes figures/tables or not. You must use Times New Roman, 12 point font size, 1.5 line spacing and with a right, left, top and bottom margin of at least 2 cm.  This template is formatted accordingly. In the "Confirmation" \ section of the application form, you must confirm that your project description observes the permitted maximum length, before you can submit your application. The Council will disregard any portions of the project description that exceed the permitted maximum length.
% Delete the instruction

% ----------------------------------------------------------------------
% Begin writing your project description

%\section{Abstract for the application}
%
%Artificial intelligence (AI) with big data is a promising technology to solve several
%seemingly unrelated problems, such as object recognition in images used for
%fruit sorting or obstacle detection in self-driving cars, voice recognition and
%natural language processing used in smartphones, or diagnostics in health car.
%Some applications of AI will be in embedded systems where resources are
%constraint and result must be delivered within a given deadline.
%
%The proposed project targets development of AI for embedded real-time systems,
%which includes also ``Internet of Things'' (IoT) devices.
%We target the AI based classification in the embedded devices with timing constraints.
%As main technology for classification we will use artificial neural networks (ANN).
%Training is usually not associated with time constraints and we envision that
%the training is performed in the cloud, and not in the embedded device.

\section{Ideas and Questions}

\begin{itemize}
\item Widex is interested in Chisel, may join a DFF proposal
\item Peter Stst + microsoft cloud into Chisel project
\item DeepSpec or DeepSp?? end-to-end for Chisel
\item WP on VHDL generation from Chisel for better verification
\item Verification (check what is current praxis)
\item Industry issue is verification: how from Chisel to VHDL/Verilog
\item How much ASIC design is done in DK? Revenue numbers?
\item There are not enough HW designers available, so they shall be more productive
\item Or we can attract SW developer with Chisel to do HW design
\item Future cloud server will include FPGA for application speedup, how to program them
\item Support letter from UCB
\item External stay at UCB
\item Richard should be part of it
\item One PhD at ITU and PhD or postdoc (Lefteris at DTU)
\item Kasper n hours per week plus maybe part time Torur
\item Chisel workshop including hands-on tutorial in DK, at IDA, have an ok letter from them
\item Thomas from Microsemi on board
\item Konstantin Vinogradov <const.vin@gmail.com> from Widex is interested in a (industrial) PhD, might be a named candidate
\item See for arguments: \url{https://cacm.acm.org/magazines/2020/7/245701-domain-specific-hardware-accelerators/fulltext}
\end{itemize}

\subsection{Contacts}

Jesper Birch <jb@napatech.com> sent a word document and is interested.

Teledyne:

"Rytter, Morten (INT)" <Morten.Rytter@Teledyne.com>

simon.andersen@teledyne.com



One option would be for us to deliver test cases and discussions of what is needed for our company. 



\subsection{TODO}

Send proposal draft proposal and of August to Jesper, Thomas, ...

\section{From InfinIT}

Digital systems are already an integral part of our life. These systems are built out of microprocessors, application specific integrated circuits (ASICs), and field-programmable gate arrays (FPGAs). Several companies in Denmark are building digital systems. To increase competitivity of those companies, we need tools and methods to increase the productivity in designing and especially testing digital systems. Compared to software development and testing, digital design and testing methods and tools lack several decades of development. Within this project we plan to leverage software development and testing methods for digital design. This project explores the hardware construction language Chisel with Scala and the Universal Verification Method (UVM) with SystemVerilog for design and test of digital systems.



UVM is becoming an industry standard for design verification. On the other hand there is an active development on a new hardware construction language, called Chisel. Chisel is embedded in Scala to write so-called hardware generators. Chisel is also called: software defined hardware. Another feature of the Chisel/Scala combination is to write models of the environment of the hardware design in Scala. As an example take a network interface (e.g., Ethernet) written as a high-level model in Scala connected to a microprocessor written in Chisel. With this example we are able to develop and test network code on the microprocessor, which is our digital design under test.



As a first step we will explore and compare the two approaches: UVM/SystemVerilog and Chisel/Scala. When we generate hardware from Chisel, we generate a Verilog description of the digital circuit. This Verilog description can further be tested within UVM (plain Verilog is valid SystemVerilog). As a next step, we will explore how test, simulation, and verification code written in Scala to develop the Chisel description of the digital circuit can be reused at the UVM level to test the generated Verilog description of the circuit.


Modern software techniques can be applied on testing where plausible. For example, fuzzing is a mature solution for producing random and yet meaningful inputs to trigger program failures. Symbolic execution explores program paths systematically via constraint solving.



We are in contact with the developers of Chisel at the University of California in Berkeley, and especially with Richard Lin, who is developing the new testing framework for Chisel. Richard is interested in this project and the integration with UVM. Therefore, we agreed to have a cooperation meeting during the project at UC Berkeley.



The project fits into the Infinit topic of IoT. The things of IoT are digital systems, often small and application specific systems. Application specific systems are built out of digital systems either with a dedicated ASIC or an FPGA.



This mini-project will be executed in close cooperation with Microchip, WSA, Synopsys, and Syosil. 



The students involved in the research project well then be well educated future engineers for digital system design and verification.



3. Aktiviteter (beskriv) 1. Learning and exploring SystemVerilog/UVM (with Synopsys)
2. Learning and exploring Chisel/Scala
3. Defining two use cases together with Microchip
4. Developing the two use-cases in Chisel and SystemVerilog with a comparison
5. Developing the verification environment including high-level models of the environment in UVM and Chisel/Scala with a comparison
6. Application of the UVM verification of the Chisel generated Verilog code
7. Scala based testing and verification on top of UVM
8. Develop an open course on verification of digital systems for DTU and use in industry




4. MilestonesKnowledge of the tools30/4/2020Definitions of the use cases31/5/2020Use cases developed30/7/2020Verification and high-level models developed31/8/2020Cross verification from Scala to UVM functional, course material finalize31/10/2020

5. Deltagere



DTU, CVR-nr. 30 06 09 46, Martin Schoeberl (project lead) (masca@dtu.dk) and Jan Madsen (jama@dtu.dk)
DTU will develop the use cases and the verification environment with UVM and Chisel/Scala. DTU will transfer knowledge on Chisel to Microchip and WSA.
ITU, CVR-nr. 29 05 77 53, Peter Sestoft (sestoft@itu.dk) and Zhoulai Fu (zhfu@itu.dk)
ITU will apply methods from software testing to digital hardware verification.
Aarhus Universitet, CVR-nr: 31119103, Farshad Moradi (moradi@eng.au.dk)
AU will explore UVM verification of Chisel generated Verilog code.
Microchip Semiconductor Corp. A/S, CVR-nr. 24224694, Thomas Aakjer (Thomas.Aakjer@microchip.com)
Microchip will provide use cases for the research in design and verification where DTU can explore Chisel with Scala.
WS Audiology Denmark A/S, CVR-nr. 40296638, Ketil Julsgaard (ketil.julsgaard@wsa.com)
WSA will provide digital-signal processing use cases for cosimulation of a Chisel description with a high-level description.
Synopsys, CVR-nr. 25600568, Martine Chegaray (Martine.Chegaray@synopsys.com)
Synopsis will provide the tools for UVM for the project and guide the usage.
Syosil Aps, CVR-nr. 29399417, Jacob Sander Andersen (jacob@syosil.com)
Syosil will support the researchers with education in using UVM.


6. Resultater og vision for 

The vision of the project is a highly productive method for designing and (more importantly) verification of digital systems by a combination of the modern hardware construction language Chisel/Scala with the industry standard UVM.
7. Videnspredning

The research work will be documented by publications and presented at relevant conferences (for example DATE), funded by other means, not by this project.

At the end of the project we will present the method at a workshop open for Danish industry in digital system design.

The new development and verification method will be used and taught in courses on digital electronics at DTU.\newpage

\section{Introduction and Objectives -- A Start}
\label{sec:objectives}

We cannot longer depend on Moor's Law to increase computing performance~\cite{dark-silicon:2011}.
Performance increase with general-purpose processors came to a halt.
The only way to achieve higher performance or lower energy consumption
is by building domain-specific hardware accelerators~\cite{domain-hw-acc:2020}.
Furthermore, the production of a chip is very expensive. Therefore, it is important to get the design right
at the first tape-out. Throughout testing and verification of the design is mandatory.

To efficient develop and verify those accelerators we can learn from software development trends
such as agile software development~\cite{agile:manifesto}.
We need agile hardware development~\cite{henn-patt:turing:2019}.

Until a few years, the two main design languages Verilog and VHDL dominated the
design and testing of digital circuits. However, both languages are decades behind
modern languages for software development.

Resent advances with SystemVerilog and Chisel bring object-oriented programming
into the digital design and verification process. SystemVerilog, as an extension of Verilog,
adds object-oriented concepts for the non-synthesizable verification code.
Chisel is a language, embedded in Scala, to describe digital circuits.
Circuits described in Chisel can be tested and verified with a Chisel testing
framework and tests written in Scala.
Scala/Chisel brings object-oriented and functional programming into the world of
digital design.

This project proposes a research project that aims at building a testing framework
in Scala that takes the best methods from UVM and from decades of experience
in software testing.
Furthermore, our aim is to build on open-source projects only. Therefore, our
work is in open-source as well.


\section{Background and State-of-the-Art}
\label{sec:background}


\paragraph{Hardware Verification}




\paragraph{Software Testing Methods}




\section{Research Plan}

\paragraph{xxx}



\paragraph*{Dissemination and Publication}

Scientific results will be published and presented at international
conferences (DATE, ...) and in relevant scientific journals.
We expect that most tasks will result in at least one publication.
One PhD theses will publish the results from the project.
We aim to publish in open access, to a large extent in the gold open access model.
However, publishers such as ACM also allow publishing in green open access
at no additional cost, where a pre-print version of a paper can be uploaded,
for example, to ArXiv.

The results from the project will be available as open-source under the
industry-friendly BSD license.
%Open-source research projects attract
%other researchers, developers, and industrial partners
%to use and build on the results of the project.
A project web site will host the project documentation, the published papers, and the
source code of the design.



\section{Practical Feasibility}

%The Embedded Systems Engineering section at DTU Compute provides
%the intellectual environment and the infrastructure (e.g., regression test server...) that we need for an ambitious research project.
%Furthermore, DTU Compute provides the infrastructure (e.g., an automatic test
%environment for regressions tests, web server).

\paragraph*{Internationalization}



\paragraph*{Industrial Cooperation}



\paragraph*{Human Resources}

For the SDH project we
request funding of the PhD student (NN) and postdoc (???).
Each of the senior researchers will contribute to the SDH research project.

%We intend to build a group with one PhD student, one postdoc, and
%two senior researchers at DTU.

%Quoted from the Diversity and Gender statement at DTU:
%``Diversity, equal treatment, and equality are integral to DTU, being an international
%university in scope and standard, and are fundamental principles underlying DTU's
%expectations of respect and equality''.
As the already named researchers are all male, we will actively search
for a female researcher for the PhD position.
However, the PhD position will be announced openly and men and women
will have equal opportunities for applying.


  {\bf Martin Schoeberl (MS)} is associate professor at DTU Compute and is the PI.
   His research interest is in computer architecture for real-time systems. During his work at Coin,
   he has developed an ANN system to detect different welding parts using
   properties, such as area, circumference, and momentum, for classification.
   Martin has worked on bringing machine learning to an embedded Java processor~\cite{pedersen:2006-64}, including a multicore
   version~\cite{jop:cmpsvn}, and exploring the WCET analysis~\cite{jop:wcet:spe}.
   
\todo{below is old stuff}
%\subsection*{Experimental Facilities}
%
%Development and simulation of the RTAI hardware can be
%performed on standard desktop PCs.
%For evaluation of individual design components small and cheap FPGA
%boards, that are already available, can be used. For the evaluation of the
%full system design of a switches with ANNs, we intend to buy three high-performance
%FPGA boards.
%%
%The needed software (e.g., VHDL simulation, FPGA compilation) is freely available.


%\vspace{-2mm}
\begin{table*}% [h!]
{\small
  \begin{center}
    \begin{tabular}{lccp{110mm}l}
      \toprule
      Task          & PM  & Person &  Description \\
      \midrule
      Recruiting  & 1 & MS and JS & Recruiting of the NN PhD \\
      \midrule
      WcetCode     &  9      & HP     & Generation of WCET analyzable code out of TF training results.\\
      MultiCore     &  6      & HP     & Speedup with mulitcore generated code using a network-on-chip and
      a shared scratchpad memory.\\
      Accelerator    &  9      & HP     & Design of a hardware accelerator and integrate with the multicore.\\
      WcetAcc     &  6      & HP     & WCET analysis of the software and accelerator implementations.\\
      \midrule
      SOA  & 3      & PhD1     &  Explore state of the art in hardware implementation of an ANN. \\
      HardNeurals  & 6      & PhD1     &  Implement hardware generators for ANN with power-of-2 weights. \\
      TFLearn  & 6      & PhD1     &  Adapt TF learning to us power-of-2 weights. \\
      WcetNeurals  & 3      & PhD1     &  WCET analysis of the hardware ANN. \\
      Switch  & 6      & PhD1     &  Implement ANN hardware in a real-time Ethernet switch. \\
      \midrule
      Explore   &    2x6 & PhD1/HP    &  Explore multicore, hardware accelerator, and ANN in hardware.\\
      Eval & 2x6 & PhD1/HP & Evaluation of the results collected with the real-time Ethernet switch. \\
      \bottomrule
    \end{tabular}
  \end{center}
%  \caption{Work packages}\label{tab:packages}
}
\end{table*}

%\vspace{-5mm}
\paragraph*{Tasks, Milestones, and Timetable}

The project is divided into several tasks.
%A researcher will be
%assigned to each task as its main developer.
%A cooperative
%working style will be encouraged so, that the experience and
%knowledge of the different team members will optimally be utilized.
For each task, the time is given in person months (PM). The Gantt chart shows
the project schedule.
For an assessment of the project's success we plan following milestones:

\textbf{M1} (Month 6): The PhD student is selected and employed.

\textbf{M2} (Month 15): Two versions of ANN are implemented: multicore and the
hardware generator.

\textbf{M3} (Month 30): All development has been finished and the different versions
can be used for exploration and evaluation of the results with the real-time switch.

\textbf{M4} (Month 42): The project has finished and a PhD thesis has been handed in.

%\vspace{-5mm}
\begin{figure*}[h!]
\centering
\begin{ganttchart}[vgrid,hgrid,bar/.style={fill=gray},
x unit=3.5mm, % horizontal squeezing
%y unit chart=5mm, % vertical squeezing
y unit title=8mm,y unit chart=4.5mm, milestone top shift=.15, milestone height=.2mm, % very tight format
title label font=\footnotesize,
bar label font=\footnotesize,
milestone label font=\footnotesize,
]{1}{42}
% labels
%\gantttitle{\textbf{\normalsize{RTAI Gantt chart}}}{42} \\
\gantttitlelist{1,...,42}{1} \\
% tasks, groups and milestones
\ganttbar[name=r1]{Recruting}{1}{6} \\
\ganttmilestone[name=m1]{Milestone 1}{6} \\
\ganttbar[name=t1]{WcetCode}{1}{9} \\
\ganttbar[name=t2]{MultiCore}{10}{15} \\
\ganttbar[name=t3]{Accelerator}{16}{24} \\
\ganttmilestone[name=m2]{Milestone 2}{15} \\
\ganttbar[name=t4]{WcetAcc}{25}{30} \\
\ganttbar[name=t5]{SOA}{7}{9} \\
\ganttbar[name=t6]{HardNeurals}{10}{15} \\
\ganttbar[name=t7]{TFLearn}{16}{21} \\
\ganttbar[name=t8]{WcetNeurals}{22}{24} \\
\ganttbar[name=t9]{Switch}{25}{30} \\
\ganttmilestone[name=m3]{Milestone 3}{30} \\
\ganttbar[name=t10]{Explore}{31}{36} \\
\ganttbar[name=t11]{Eval}{37}{42} \\
\ganttmilestone[name=m4]{Milestone 4}{42}

% relations
\ganttlink{r1}{m1}
\ganttlink{t1}{t2}
\ganttlink{t2}{m2}
\ganttlink{m2}{t3}
\ganttlink{t3}{t4}
\ganttlink{t4}{m3}

\ganttlink{m1}{t5}
\ganttlink{t5}{t6}
\ganttlink{t6}{m2}
\ganttlink{m2}{t7}
\ganttlink{t7}{t8}
\ganttlink{t8}{t9}
\ganttlink{t9}{m3}
\ganttlink{m3}{t10}
\ganttlink{t10}{t11}
\ganttlink{t11}{m4}

\end{ganttchart}

\caption{The Gantt chart of RTAI}\label{fig:gantt}
\end{figure*}




%\vspace{2mm}

\newpage
\small
\bibliographystyle{abbrv}
%\bibliography{myown,jsp,noc,misc,msbib}
\bibliography{../msbib}

\end{document}
