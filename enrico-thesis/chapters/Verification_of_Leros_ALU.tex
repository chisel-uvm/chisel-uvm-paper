%% This is free and unencumbered software released into the public domain.

%% Anyone is free to copy, modify, publish, use, compile, sell, or
%% distribute this software, either in source code form or as a compiled
%% binary, for any purpose, commercial or non-commercial, and by any
%% means.

%% In jurisdictions that recognize copyright laws, the author or authors
%% of this software dedicate any and all copyright interest in the
%% software to the public domain. We make this dedication for the benefit
%% of the public at large and to the detriment of our heirs and
%% successors. We intend this dedication to be an overt act of
%% relinquishment in perpetuity of all present and future rights to this
%% software under copyright law.

%% THE SOFTWARE IS PROVIDED "AS IS", WITHOUT WARRANTY OF ANY KIND,
%% EXPRESS OR IMPLIED, INCLUDING BUT NOT LIMITED TO THE WARRANTIES OF
%% MERCHANTABILITY, FITNESS FOR A PARTICULAR PURPOSE AND NONINFRINGEMENT.
%% IN NO EVENT SHALL THE AUTHORS BE LIABLE FOR ANY CLAIM, DAMAGES OR
%% OTHER LIABILITY, WHETHER IN AN ACTION OF CONTRACT, TORT OR OTHERWISE,
%% ARISING FROM, OUT OF OR IN CONNECTION WITH THE SOFTWARE OR THE USE OR
%% OTHER DEALINGS IN THE SOFTWARE.

%% For more information, please refer to <https://unlicense.org>
%%
\chapter{Verification of Leros ALU}\label{verificationalu}

This section shows the technique illustrated in the previous chapters for
functionally verify the ALU of Leros \cite{online:leros}. The ALU showed in this
section is a slightly modified Leros ALU taken from the Github ``chiselverify"
repository \cite{github:chiselverify} in which the ``io" port of the ALU was
replaced by ``input" and ``output" ones by using a \mints{MultiIOModule} instead
of a normal \mints{Module} and using a word size of 16-bits. The Verilog version
of this ALU was already functionally verified in SystemVerilog/UVM by Kasper
Hesse, and the source code used for the SystemVerilog verification can be found
in \cite{github:chiselverify}.

The Leros ALU is a more advanced but still functionally simple ALU compared to
the one used in the previous chapters. This ALU is an accumulator based ALU that
supports eight types of operations. As for the input ports, the ALU has an
operation input called ``op," a data input called ``din," and an enable bit
called ``ena." For the output ports, the ALU has only one port called ``accu."
This output port is the result of the operation ``op," between the current
``din" input and the accumulator value if the ``ena" bit is enabled. Otherwise,
the output is the value stored into the accumulator itself. While the data input
and output width is variable and depends on the overall design specifications
(in this case 16-bits), the operation input and enable bit are fixed to 4 bits
and 1 bit, respectively. The ALU is a sequential circuit in which all the
operations take precisely one cycle, and the state is saved in the accumulator.
The operations supported are displayed in table \ref{table:leros:op}.

\begin{table}[!h]
\centering
\begin{tabular}{c|c|c|c|c|c|c|c}
 \textbf{nop} & \textbf{add} & \textbf{sub} & \textbf{and} & \textbf{or} & \textbf{xor} & \textbf{ld} & \textbf{shr}\\
 \hline
no-operation & addition & subtraction & and & or & xor & load & shift-right\\
\end{tabular}
\caption{Leros ALU opcodes.}
\label{table:leros:op}
\end{table}


The paragraph above can be considered the functional specification document for
the Leros ALU. From these specifications, it is possible to extract all the
features that need to be tested to functionally verify this ALU. From the
specifications, and assuming that the ``din" and ``accu" ports' width of the ALU
are $n$-bits, the total state space of this design is:

\begin{equation*}
   S_T = \text{inputs} \cdot \text{outputs} \cdot \text{internal states} =  2^{n} \cdot 2^1 \cdot 2^8 \cdot 2^n \cdot 2^n = 2^{3n + 9}.
\end{equation*}

Luckily, not all the states of the state space are reachable. For example, if
the ``ena" bit is set to false, all the other input parameters do not influence
the output. Thus, the amount of the state space is reduced by a large factor.

Nonetheless, this shows that even for simple design, the total state space can
be vast and hard to traverse completely. Since verifying all the combinations of
inputs for the current design is time-consuming and unnecessary for this design,
the verification plan will list only the high priority features. From the
specifications, the most critical features to verify are all eight types of
operations. These operations need to be checked for both the ``ena" bit high and
low. Secondly, all the operations have to be checked for a variety of input
values. Table \ref{table:verificationplan} shows the verification plan for this
specific ALU.


\begin{table}[!h]
\centering
\begin{tabular}{l|l|l}
\textbf{Section} & \textbf{Description} & \textbf{Coverage Type}\\
\hline
\textbf{nop} Reuslts & Check correctness of the nop operation & Test reuslts\\
\textbf{add} Reuslts & Check correctness of the add operation & Test reuslts\\
\textbf{sub} Reuslts & Check correctness of the sub operation & Test reuslts\\
\textbf{and} Reuslts & Check correctness of the and operation & Test reuslts\\
\textbf{or} Reuslts & Check correctness of the or operation & Test reuslts\\
\textbf{xor} Reuslts & Check correctness of the xor operation & Test reuslts\\
\textbf{ld} Reuslts & Check correctness of the ld operation & Test reuslts\\
\textbf{shr} Reuslts & Check correctness of the shr operation & Test reuslts\\
\textbf{op} Types & All the operations are executed & CoverPoint\\
\textbf{ena} port & Check enable/disable functionality & CoverPoint\\
\textbf{din} input & Check din with values between 0 to 0xF & CoverPoint\\
\textbf{din} input & Check din with values between 0xF to 0xFF & CoverPoint\\
\textbf{din} input & Check din with values between 0xFF to 0xFFF & CoverPoint\\
\textbf{din} input & Check din with values between 0xFFF to 0xFFFF & CoverPoint\\
\end{tabular}
\caption{Verification plan for the Leros ALU.}
\label{table:verificationplan}
\end{table}


After creating the verification plan, it is possible to proceed with the
creation of the testbench. The testbench will be based around the concept of
Coverage Driven Verification, in which the machine can automatically generate
tests, and the testbench automatically collects the coverage data for each test
created. To automatically create new tests, the machine needs to have an input
that distinguishes one test from the other. In this specific case, a list of
seeds will be passed to the machine, and for each seed in the list, the machine
will generate a new test. Since the machine will use the seeds in the list to
randomize the input data, each test's result is entirely reproducible.

As explained in the previous chapters, when the Coverage Driven Verification and
Constrained Random Verification are used, the testbench needs to generate a
sequence of meaningful transactions for the current device under test. From the
author's perspective, and considering the Chisel and ChiselTest capabilities, a
transaction is better expressed as a Chisel literal \mints{Bundle}. This choice
was deliberately made to take advantage of the more advanced Chisel
functionalities and reduce the overall verification code. This is also why the
ALU was modified and represented as a \mints{MultiIOModule} instead of a simple
\mints{Module} as shown in listing \ref{listing:example:multiIOleros}.

\begin{code}
\begin{minted}
[
frame=lines,
framesep=2mm,
baselinestretch=1.2,
fontsize=\footnotesize,
linenos
]
{scala}
class AluAccuInput(val size: Int) extends Bundle{
  val op = UInt(3.W)
  val din = UInt(size.W)
  val ena = Bool()
}

class AluAccuOut(val size: Int) extends Bundle {
  val accu = UInt(size.W)
}

abstract class AluAccuMutliIO(sizen: Int) extends MultiIOModule {
  val input = IO(Input(new AluAccuInput(sizen)))
  val output = IO(Output(new AluAccuOut(sizen)))
}

\end{minted}
\caption{Leros ALU with MultiIO interface.}
\label{listing:example:multiIOleros}
\end{code}

Other than taking advantage these advanced features of Chisel and ChiselTest,
modeling transactions as literal \mints{Bundles} implies that each transaction
carries information about the width used for the ``din" port, which will be
useful when designing the reference model.

After deciding how to represent a transaction, it is possible to create the
random class that models this transaction and randomizes it. For this specific
example, it was decided to extend the class \mints{RandObj} instead of using the
more experimental \mints{RandBundle} class. The object that captures a random
transaction is called \mints{AluTransaction}. This object, like in the ALU
specifications, has three random fields, ``din," ``op," ``ena," and one
constructor argument that represents the size of the ALU. After having modeled
the random object, and since the testbench uses a literal \mints{Bundle} as
transactions, the object defines a method called \mints{toBundle}, which, as the
name says, converts the randomized fields of the object into a randomized
transaction.

As it is, the random object captures all the possible input values for the Leros
ALU. However, this implies that a randomized transaction has the same
probability of having the ``ena" bit enabled or disabled, thus leading to the
creation of a sequence of transactions that are not very meaningful from a
functional verification point of view. To avoid repetitive sequence of
transactions with the enable bit set to low, it is possible to add a dist
constraint to the \mints{AluTransaction} class, limiting the transaction with
the ``ena" bit set to low for only 1\% of the overall random transactions. The
same process can be applied to the operation input ``op" and the ``nop"
operation. Since this operation does not produce meaningful results, it is
possible to limit the probability of producing such an operation in a random
transaction by assigning a probability of 10\% compared to the other type of
transactions.

\begin{code}
\begin{minted}
[
frame=lines,
framesep=2mm,
baselinestretch=1.2,
fontsize=\footnotesize,
linenos
]
{scala}
class AluTransaction(seed: Int, size: Int) extends RandObj {
  currentModel = new Model(seed)
  val max = pow(2, size).toInt
  val op = new Rand("op", 0, 7)
  val din = new Rand("din", 0, max)
  val ena = new Rand("ena", 0, 1)

  def toBundle: AluAccuInput = {
    new AluAccuInputPort(size).Lit(_.op -> op.value().U,
        _.din -> din.value().U,
        _.ena -> ena.value().B)
  }
  
  val values = din dist (
    0 to 0xF := 1,
    0xF to 0xFF := 1,
    0xFF to 0xFFF := 1,
    0xFFF to 0xFFFF := 1,
  )

  val onlyHigh = din dist (
    0xFF to 0xFFF := 1,
    0xFFF to 0xFFFF := 10,
  )
  
  val fewNops = op dist (
    0 := 10,
    (1 to 7) := 90,
  )

  val enaHigh = ena dist (
    1 := 99,
    0 := 1
  )

  onlyHigh.disable()
}
\end{minted}
\caption{Random Transaction Object for Leros ALU Verification.}
\label{listing:example:randobj}
\end{code}

After laying out a random transaction definition, it is necessary to create a
component that can generate these transactions. Generally speaking, in the
verification/UVM jargon, this component takes the name of the sequencer. Given
the extensive standard library provided by the Scala language, the sequencer
functionalities can be easily replaced in Scala by functions like \mints{map} or
\mints{foreach} shown in listing \ref{listing:example:sequencer}.

\begin{code}
\begin{minted}
[
frame=lines,
framesep=2mm,
baselinestretch=1.2,
fontsize=\footnotesize,
linenos
]
{scala}
val lenght = 5000
val masterT = new AluTransaction(seed, size)
val transactions: Seq[AluAccuInputPort] = (0 to lenght) map { _ =>
  masterT.randomize
  masterT.toBundle
}
\end{minted}
\caption{Random Transaction generator.}
\label{listing:example:sequencer}
\end{code}
As was said before, each transaction list is randomized using a seed passed as
an argument to the current test, making them reproducible.

After having set the infrastructure for generating random transactions, it is
possible to design the reference model. Since Chisel is hosted on top of the
Scala language, the natural choice is to develop the reference model in the
hosting language itself. Another important consideration before developing the
model is that this specific ALU lends itself to a sequential type of testing,
where an input is sent to it, and after a clock cycle, the output is computed.
Following these considerations, the reference model was developed as a singleton
object containing five different functions listed in
\ref{listing:referencemodel}. The essential functions of this model are
\mints{compare} and \mints{generateTransactions}. This last function has a
signature of type:
\begin{code}
\begin{minted}
[
frame=lines,
framesep=2mm,
baselinestretch=1.2,
fontsize=\footnotesize,
linenos
]
{scala}
def generateTransactions(listT: List[AluAccuInputPort], accu: AluAccumulator):
List[AluAccuOutPort]
\end{minted}
\caption{generateTransactions signature.}
\label{listing:gentransactionsinga}
\end{code}

Here \mints{AluAccumulator} is a case class with only one field \mints{value}
representing the current value of the accumulator. As shown by its signature,
this function takes a list of input transactions and an accumulator and
generates a list of output transactions. As an example, if the reset value of
the ALU accumulator is zero, it is possible to simulate its behavior by calling
the \mints{generatedTransactions} method with parameters the list of
transactions to test and the \mints{Accumulator(0)}. Modeling the golden-model
and the accumulator as two distinct entities allows the separation of the data
from the functionality, and thus it allows the creation of a ``pure" functional
golden-model that can be used in parallel tests.

\begin{code}
\begin{minted}
[
frame=lines,
framesep=2mm,
baselinestretch=1.2,
fontsize=\footnotesize,
linenos
]
{scala}
object AluGoldenModel {
  def genNextState(transaction: AluAccuInputPort, accu: AluAccumulator):
  AluAccumulator

  def genOutputStates(listT: List[AluAccuInputPort], accu: AluAccumulator):
  List[AluAccumulator]
  
  def serializeAccu(accu: AluAccumulator, size: Int): AluAccuOutPort

  def generateAluAccuTransactions(listT: List[AluAccuInputPort],
    accu: AluAccumulator): List[AluAccuOutPort]

  def compareSingle(transaction: (AluAccuOutPort, AluAccuOutPort)): Boolean

  def compare(transactions: List[(AluAccuOutPort, AluAccuOutPort)]): List[Boolean]
}
\end{minted}
\caption{Leros ALU reference model.}
\label{listing:referencemodel}
\end{code}


The list of transactions generated by the golden-model can then be compared to
the list of transactions generated by the ALU itself using the \mints{compare}
method provided by the \mints{AluGoldenModel} object. The \mints{compare} method
uses the API offered by ScalaTest to compare the value produced by the ALU and
the golden-model. Also, it is essential to note that by using this method, the
ALU is tested for all the transactions in the test, and it does not merely stop
the test at the first miscomputed result.


When the golden-model is completed, it is time to create the testbench itself.
Given the power of Scala and ScalaTest, the testbench can be modeled in
different ways and shapes. For this functional verification, and based on the
author's preferences, the testbench was modeled by creating a trait containing
the behavior functionality of the ALU. Compared to the traditional testbench
written in ChiselTest, this testbench allows the creation of a specific test
result for each seed used in the testing. The trait called \mints{ALUBehaviour}
contains a function called \mints{compute}. The function has three parameters.
The first one is the name of the test. The name is needed because, since each
time we call the function \mints{compute}, ScalaTest generates a new test, it is
a requirement that the name of the test be unique, so in this case, each test is
differentiated by the value of the seed with which it is executed. The second
parameter is the size of the current ALU to test, and the last one is the list
of random transactions to test. As the name says, the function simulates the
ALU, collects the list of transactions generated by it, and compares them to the
one generated by the golden-model.

\begin{code}
\begin{minted}
[
frame=lines,
framesep=2mm,
baselinestretch=1.2,
fontsize=\footnotesize,
linenos
]
{scala}
trait AluBehavior {
  this: FlatSpec with ChiselScalatestTester =>

  def compute(seed: Int, size: Int, inputT: List[AluAccuInputPort]): Unit = {
    it should s"test alu with seed = $seed" in {
      test(new AluAccuMultiChisel(size)) { dut =>
        val computedTransactions: List[AluAccuOutPort] = inputT.map { T =>
          dut.input.poke(T)
          dut.clock.step()
          dut.output.peek
        }
        val goldenTransactions =
            AluGoldenModel.generateAluAccuTransactions(inputT, AluAccumulator(0))
        AluGoldenModel.compare((computedTransactions zip goldenTransactions))
      }
    }
  }
}
\end{minted}
\caption{ALU Behavior trait.}
\label{listing:alubehaviourtrait}
\end{code}


After completing the testbench, the last step is to insert the coverage probe to
collect the various test executions' coverage data. The first test probe to be
inserted from the verification plan is a \mints{CoverPoint} for collecting the
data from the operation input. Furthermore, it is essential to check that all
the operations work correctly with both the ``ena" bit high or low. For this
reason, a \mints{CrossPoint} between the ``ena" bit and the operation mode was
added as a functional probe. Lastly, it is necessary to check that the ALU works
for all ranges of values, from the lower values to 0xFFFF. In this case, no
internal probe for the accu register was added because this type of probe is not
supported in the current version of ``chiselverify." It was decided to add a
coverage probe to the ALU's output even though the result is directly related to
the input values of the ALU and checked during testing with the golden-model.
Listing \ref{listing:aluprobes} shows how the functional coverage probe are
declared inside the testbench.
\begin{code}
\begin{minted}
[
frame=lines,
framesep=2mm,
baselinestretch=1.2,
fontsize=\footnotesize,
linenos
]
{scala}

val cr = new CoverageReporter(dut)
cr.register(
  List(
  CoverPoint(dut.input.op , "op")(
    List(
      Bins("nop", 0 to 0), Bins("add", 1 to 1),
      Bins("sub", 2 to 2), Bins("and", 3 to 3),
      Bins("or", 4 to 4), Bins("xor", 5 to 5),
      Bins("ld", 6 to 6), Bins("shr", 7 to 7))),
    [...]
    [...]
    CrossPoint("operations cross enable", "op", "ena")(
      List(
        CrossBin("operation enable", 0 to 7, 1 to 1),
        CrossBin("operation disabled", 0 to 7, 0 to 0),
    )))
)
\end{minted}
\caption{Functional coverage probes for the Leros ALU.}
\label{listing:aluprobes}
\end{code}

After inserting all the coverage probes into the testbench, it is possible to
let the testbench run until the coverage quality is reached by adding new seeds
to the seed list. After adding twelve seeds and inserting a new type of test
that tests only values between 0xFF and 0xFFFF, the result of the coverage
report 100\% for all the CoverPoints as shown in listing
\ref{listing:alureport}. In total, the testbench executed 24 tests generated by
12 different seed ($2 \times 12$) with 5000 transactions each for a total of
120000 random transactions.

\begin{code}
\begin{minted}
[
frame=lines,
framesep=2mm,
baselinestretch=1.2,
fontsize=\footnotesize,
linenos
]
{bash}
============ COVERAGE REPORT ============
============== GROUP ID: 1 ==============
COVER_POINT PORT NAME: op
BIN nop COVERING Range 0 to 0 HAS 24 HIT(S) = 100.00%
BIN add COVERING Range 1 to 1 HAS 24 HIT(S) = 100.00%
BIN sub COVERING Range 2 to 2 HAS 24 HIT(S) = 100.00%
BIN and COVERING Range 3 to 3 HAS 24 HIT(S) = 100.00%
BIN or COVERING Range 4 to 4 HAS 24 HIT(S) = 100.00%
BIN xor COVERING Range 5 to 5 HAS 24 HIT(S) = 100.00%
BIN ld COVERING Range 6 to 6 HAS 24 HIT(S) = 100.00%
BIN shr COVERING Range 7 to 7 HAS 24 HIT(S) = 100.00%
=========================================
COVER_POINT PORT NAME: din
BIN 0xF COVERING Range 0 to 15 HAS 180 HIT(S) = 100.00%
BIN 0xFF COVERING Range 15 to 255 HAS 2868 HIT(S) = 100.00%
BIN 0xFFF COVERING Range 255 to 4095 HAS 17970 HIT(S) = 100.00%
BIN 0xFFFF COVERING Range 4095 to 65535 HAS 67555 HIT(S) = 100.00%
=========================================
COVER_POINT PORT NAME: accu
BIN 0xF COVERING Range 0 to 15 HAS 300 HIT(S) = 100.00%
BIN 0xFF COVERING Range 15 to 255 HAS 3033 HIT(S) = 100.00%
BIN 0xFFF COVERING Range 255 to 4095 HAS 14288 HIT(S) = 100.00%
BIN 0xFFFF COVERING Range 4095 to 65535 HAS 66163 HIT(S) = 100.00%
=========================================
COVER_POINT PORT NAME: ena
BIN disabled COVERING Range 0 to 0 HAS 24 HIT(S) = 100.00%
BIN enabled COVERING Range 1 to 1 HAS 24 HIT(S) = 100.00%
=========================================
CROSS_POINT operations cross enable FOR POINTS op AND ena
BIN operation enable COVERING Range 0 to 7 CROSS Range 1 to 1 HAS 192 HIT(S)
                                                                        = 100.00%
BIN operation disabled COVERING Range 0 to 7 CROSS Range 0 to 0 HAS 192 HIT(S)
                                                                        = 100.00%
=========================================
=========================================
[info] AluVerification:
[info] AluAccumulator
[info] - should test alu with seed = 30, and normal values
[info] - should test alu with seed = 104, and normal values
[info] - should test alu with seed = 60, and normal values
[info] - should test alu with seed = 90, and normal values
[info] - should test alu with seed = 200, and normal values
[info] - should test alu with seed = 50, and normal values
[info] - should test alu with seed = 22, and normal values
[info] - should test alu with seed = 2000, and normal values
[info] - should test alu with seed = 40, and normal values
[info] - should test alu with seed = 900, and normal values
[info] - should test alu with seed = 70, and normal values
[info] - should test alu with seed = 23, and normal values
[info] - should test alu with seed = 30, and high values
[info] - should test alu with seed = 104, and high values
[info] - should test alu with seed = 60, and high values
[info] - should test alu with seed = 90, and high values
[info] - should test alu with seed = 200, and high values
[info] - should test alu with seed = 50, and high values
[info] - should test alu with seed = 22, and high values
[info] - should test alu with seed = 2000, and high values
[info] - should test alu with seed = 40, and high values
[info] - should test alu with seed = 900, and high values
[info] - should test alu with seed = 70, and high values
[info] - should test alu with seed = 23, and high values
\end{minted}
\caption{Coverage report of Leros ALU.}
\label{listing:alureport}
\end{code}

\section{Discussion}
As explained in the previous chapters, functional coverage is not the only type
of coverage to be considered when verifying a model. The verification quality
needs to include code coverage, state machine coverage, and so on, which are
currently not provided by the ``chiselverify" library. Another critical remark
is about not throwing away the coverage metrics collected during the
verification process. Currently, in ``chiselverify" the coverage metrics are
only contained and accessible in the test suite, thus making it harder to rely
on the library for more significant projects.
 
Accelera has an open specification for the Unified Coverage Interoperability
Standard (UCIS) \cite{coveragedatabaseucis}. UCIS, as the name says, is an open
standard for storing the coverage results and is currently used by the leading
proprietary tools. Integrating this standard into the library will allow the
cooperation between proprietary tools and Chisel and, more importantly, store
the various tests' coverage results to be further analyzed or as a historical
reference.


\subsection{Future Work}
As for now, the library with the JaCoP backend implements most of the constraint
syntax and language present in SystemVerilog. The \mints{RandObj} class is
reliable, and the JaCoP solver offers equivalent performance compared to the
ones offered by proprietary tools. Compared to SystemVerilog, the JaCoP backend
is missing ``soft" constraint and constraint evaluation order. These two
features are useful but not essential, as shown by the Github research.

From the author's perspective, this library represents a starting point for
experimenting with more advanced features of Chisel and Constraint Random
Verification. A possible addition to the library would be to extend the
library's capabilities to create reliably nontrivial random literal
\mints{Bundles} and move these features from the ``experimental" package to the
JaCoP one. Moreover, it would be relevant and practical to implement
\mints{RandObj} and \mints{RandBulde}, random class translator, to Minizinc
\cite{nethercote2007minizinc} problems and solve them with one of the many
minizinc solvers.

Another possible future development would be to improve the interface by masking
the various backends' limitations. For example, for the JaCoP backend, the
interface should be implemented so that random fields with a bit-width of more
than 29 are automatically split into two random variables. While modifying it,
the interface should also be refactored into a more ``agnostic" design to allow
runtime switching between different backends.
