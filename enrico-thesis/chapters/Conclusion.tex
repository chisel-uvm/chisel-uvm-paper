%% This is free and unencumbered software released into the public domain.

%% Anyone is free to copy, modify, publish, use, compile, sell, or
%% distribute this software, either in source code form or as a compiled
%% binary, for any purpose, commercial or non-commercial, and by any
%% means.

%% In jurisdictions that recognize copyright laws, the author or authors
%% of this software dedicate any and all copyright interest in the
%% software to the public domain. We make this dedication for the benefit
%% of the public at large and to the detriment of our heirs and
%% successors. We intend this dedication to be an overt act of
%% relinquishment in perpetuity of all present and future rights to this
%% software under copyright law.

%% THE SOFTWARE IS PROVIDED "AS IS", WITHOUT WARRANTY OF ANY KIND,
%% EXPRESS OR IMPLIED, INCLUDING BUT NOT LIMITED TO THE WARRANTIES OF
%% MERCHANTABILITY, FITNESS FOR A PARTICULAR PURPOSE AND NONINFRINGEMENT.
%% IN NO EVENT SHALL THE AUTHORS BE LIABLE FOR ANY CLAIM, DAMAGES OR
%% OTHER LIABILITY, WHETHER IN AN ACTION OF CONTRACT, TORT OR OTHERWISE,
%% ARISING FROM, OUT OF OR IN CONNECTION WITH THE SOFTWARE OR THE USE OR
%% OTHER DEALINGS IN THE SOFTWARE.

%% For more information, please refer to <https://unlicense.org>
%%
\chapter{Conclusion}\label{conclusion}
This thesis presents the work of implementing a library for random constraint
verification for Chisel. After introducing the concept of functional
verification, its main advantages and comparing it to the traditional
techniques, the document focuses on the constraint random verification branch of
functional verification. Here, the concept of constraint programming and the
theory behind it is introduced and explained. Subsequently, F-CSP is introduced
as a Scala functional library for constraint random verification. This library,
developed from the ground up, showed its limitations, and it was abandoned
during the course of the project. Learned from the implementation mistakes of
F-CSP, Chisel-CRV was the outcome of the combination between JaCoP, a library
for constraint programming written in Java, and a DSL layer for specifying
constraint in a verification environment. This library, inspired mainly by
SystemVerilog constraint API, successfully implements most of them. Finally, the
last part of the document describes the library's usage by functionally verify
the Leros ALU using coverage driven verification.


To conclude, this thesis shows how ductile is the Chisel language and how it can
be quickly extended to add functionalities that until now were provided only by
proprietary tools. Moreover, with the Leros ALU verification, this thesis shows
that the new functionalities added are, thanks to the Scala malleability, more
elegant and produced less boilerplate code compared to a similar verification in
SystemVerilog.

What is currently missing for Chisel is a more consistent environment where
these open-source tools are developed following a set of shared interfaces and
standards to make them reusable with other development tools. From the author's
perspective, the online community must agree on standardizing the functional
verification libraries of Chisel and unify the development efforts to accelerate
this hardware description language's verification capabilities.

Other than language directives like constraint programming or functional
coverage, currently, what Chisel is currently missing are tools that help
verification engineers reason about functional verification. One of the
significant strengths of proprietary development environments is tools that can
automatically generate coverage reports or collect the coverage and express it
in terms of verification quality, which are critical for large-sized projects.
