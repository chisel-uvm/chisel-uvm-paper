% DFF template for Latex and A4 paper.
% 12pt Times New Roman on 1.5 line spacing and 2 cm margins.

% ----------------------------------------------------------------------

% Either format with
%    pdflatex projectdescription.tex
% Or if you use dvips and ps2pdf, remember to specify A4 paper:
%    latex  projectdescription
%    dvips  -ta4 projectdescription -o projectdescription.ps
%    ps2pdf -sPAPERSIZE=a4 projectdescription.ps

% ----------------------------------------------------------------------

\documentclass[fleqn,12pt]{article}
\usepackage[a4paper,top=2cm,bottom=2cm,left=2cm,right=2cm]{geometry}
\usepackage{times}
\usepackage[danish,english]{babel}
\usepackage[utf8]{inputenc}
\usepackage[T1]{fontenc}
% \usepackage{graphicx}         % For PDF figures
% \usepackage[dvips]{graphicx}  % For EPS figures, using dvips + ps2pdf

\usepackage[colorlinks=true,linkcolor=black,citecolor=black]{hyperref}
\usepackage{booktabs}

\usepackage{tikz}
\usetikzlibrary{positioning,fit}
\usetikzlibrary{shapes,backgrounds}
\usetikzlibrary{arrows,fit,automata,positioning,decorations,calc}
\usetikzlibrary{spy}
\usetikzlibrary{matrix,chains,decorations.pathreplacing}
\usepackage{pgfgantt}

\newcommand{\code}[1]{{\textsf{#1}}}

% Adding comments in the text during writing process
\newcommand{\todo}[1]{{\it TODO: #1}}
\newcommand{\note}[1]{{\it Note: #1}}
\newcommand{\martin}[1]{{\color{blue} Martin: #1}}
\newcommand{\jens}[1]{{\color{green} Jens: #1}}

% uncomment following for final submission
%\renewcommand{\todo}[1]{}
%\renewcommand{\note}[1]{}
%\renewcommand{\martin}[1]{}
%\renewcommand{\jens}[1]{}




%%% ZF
\usepackage{listings}
\lstset{
	columns=fullflexible,
	%        basicstyle=\ttfamily\footnotesize,
	basicstyle=\ttfamily\small,      
	%columns=fullflexible, keepspaces=true,
	numbers=left,    
	numberblanklines=false,
	captionpos=b,
	%	breaklines=true,
	escapeinside={@}{@},
	numbersep=5pt,
	language=C,
	tabsize=2,
	breakatwhitespace=true,
	breaklines=true,
	deletekeywords={for},
	%        keywordstyle=\ttfamily
	numbersep=5pt,
	xleftmargin=.10in,
	%xrightmargin=.25in
}



\begin{document}
% Empirically this seems to match MS Word's idea of 1.5 line spacing.
% DO NOT CHANGE
\setlength{\baselineskip}{1.44\baselineskip}

% ----------------------------------------------------------------------
% Enter the title of the project and your name

\begin{center}
  {\LARGE\bf DIREC Project Description }\\[1ex]
  %https://dtudk.zoom.us/j/61032875012
  {\LARGE\bf Hardware as a Service (HaaS)}\\[1ex]
  {\large Martin Schoeberl, DTU Compute}\\[1ex]
 \end{center}

% ----------------------------------------------------------------------
% Delete the instruction

%\noindent
%The length of the project description must not exceed the number of pages indicated for the specific instrument in the Call, excl. a brief list of references, whether it includes figures/tables or not. \emph{You must use Times New Roman; 12 point font size; 1.5 line spacing; and with a right, left, top and bottom margin of at least 2 cm. The requirements for font size also apply to any footnotes, tables and figure texts.}  This template is formatted accordingly. In the "Confirmation" \ section of the application form, you must confirm that your project description observes the permitted maximum length, before you can submit your application. The Council will disregard any portions of the project description that exceed the permitted maximum length. 
%
%Please note that the summary of the project should only be included in the application form (under the section "Title and scientific content"). The summary should therefore not be included in the project description.
% Delete the instruction

% ----------------------------------------------------------------------
% Begin writing your project description

%\section{Abstract for the application -- do not include in this document}
%
%Performance increase with general-purpose processors has come to a halt. We can no longer depend on Moore's Law to increase computing performance. The only way to achieve higher performance or lower energy consumption is by building domain-specific hardware accelerators. These accelerators can be built in ASICs or in FPGAs in the cloud. To efficiently design and verify those domain-specific accelerators, we need agile hardware development.
%
%
%This project aims to develop a method and concrete tools for agile hardware development. We will use tools, languages, development, and testing methods from the last decades in software development and apply them to hardware design. We aim to raise the tooling level for a digital design to increase productivity. Time for verifying (testing) of digital systems is about double the time of developing them in the first place. Therefore, this project's central focus is on applying software development testing methods to hardware development.
%
%\subsection{Popular Abstract}
%
%Digital systems are a central part of our current and future digital enhanced live. Denmark has a considerable industry in the design and development of digital systems. However, designing and verifying such digital systems becomes an ever-growing challenge. The main issue is the usage of old tools compared to tools for programming those devices. This project aims to apply tools and methods from software development to hardware development to increase productivity. Danish firms will benefit from the results of this project.
\begin{abstract}
  Digital systems are a central part of our current and future digital enhanced live. Denmark has a considerable industry in the design and development of digital systems.\todo{such as...?} However, designing and verifying such digital systems becomes an ever-growing challenge. The project addresses: (1) how to make hardware accelerates in the cloud accessible for software developers,\todo{\#1 grammar issue; maybe rephrase?} (2) how hardware verification can learn from software testing, and (3) co-verification of software and hardware with modern tooling.

  This project requests funding for two PhD students at DTU and ITU.
\end{abstract}

\section{Introduction and Objectives}
\label{sec:objectives}

\todo{Mix and merge text from the first phase document.}

We can no longer depend on Moore's Law to increase computing performance~\cite{dark-silicon:2011}.
Performance increase with general-purpose processors has come to a halt.
The only way to achieve higher performance or lower energy consumption
is by building domain-specific hardware accelerators~\cite{domain-hw-acc:2020}.
These accelerators can be built in chips or in FPGAs in the cloud.
The production of a chip is costly. Therefore, it is essential to get
the design right at the first tape-out. Thorough testing and verification of the design is mandatory.

To efficiently develop and verify those accelerators, we can learn from software development trends such as agile software development~\cite{agile:manifesto}.
We believe that {\bf we need to adapt to agile hardware development}~\cite{henn-patt:turing:2019}.

Furthermore, as accelerators become part of the cloud service, i.e., FPGAs in the cloud,
software developers will increasingly need to adapt critical algorithms to FPGAs to enhance performance.
Hence, it is imperative to make accelerator design accessible for software developers.
By adapting hardware accelerator design to the methods and tools of contemporary software design,
it is possible to bridge both domains catering for a more uniform hardware/software development process.

Until a few years, the two main design languages Verilog and VHDL dominated the
design and testing of digital circuits.
%However, both languages are decades behind
%modern languages for software development.
However, compared to software development and testing, digital design and testing methods
and tools lack several decades of development. Within this project, we plan to
{\bf leverage software development and testing methods for digital design.}
This project explores the hardware construction language Chisel~\cite{chisel:dac2012} with Scala
and the Universal Verification Method (UVM) with SystemVerilog~\cite{SystemVerilog} for
the design and test of digital systems.

This project aims to develop a method and concrete tools for agile hardware development.
We will use tools, languages, development, and testing methods from the last decades in
software development and apply them to hardware design.
We aim to {\bf raise the tooling level for a digital design to increase productivity}.
Time for verifying (testing) of digital systems is about double the time of developing
them in the first place.
Therefore, this project's central focus is on {\bf applying software
testing methods for hardware testing}.

We will build a combination of open-source tools for verifying
circuits described in mixed languages (VHDL, SystemVerilog, and Chisel).
It builds on top of the Chisel hardware construction language and uses Scala to drive the verification. 
%We will explore the testing strategy used in UVM in the context of verifying hardware described in Chisel.

%This project proposes a research project that aims at building a testing framework
%in Scala that takes the best methods from UVM and from decades of experience
%in software testing.
%The developed framework shall support mixed languages (VHDL, SystemVerilog, and Chisel)
%to be able to integrate legacy code.
%Furthermore, our aim is to build on open-source projects. Therefore, our
%work will be in open-source as well.

%\begin{itemize}
%\item Maybe more ideas: \url{https://www.youtube.com/watch?v=dbOi_Gboi_0}, \url{https://www.youtube.com/watch?v=4FCZLrauDcE}
%\item Higher-Order Hardware Design, meta-programming language and the actual hardware construction language are the same, usually Python or Perl scripts with strings
%\item Have a measurable objective (LoC UVM vs Scala, SystemVerilog vs Chisel)
%\item Chisel has all the Scala/Java tooling and libraries available, SystemVerilog is a niche language
%\item Rise level of tooling, not necessarily level of abstraction in HW description
%\item IDE
%\item Namespace with packets make it easier to combine IPs
%\item Industry issue is verification: how from Chisel to VHDL/Verilog
%\item How much ASIC design is done in DK? Revenue numbers?
%\item There are not enough HW designers and verification engineers available, so they shall be more productive
%\end{itemize}

\section{Background and State-of-the-Art}
\label{sec:background}

\todo{Some more is needed here. Stuff that we could also need for our paper.}

\todo{Ref our initial ``towards'' paper.}

%\begin{itemize}
%\item Verification (check what is current praxis)
%\item cocotb
%\item See pull request for ref to constraint random generation
%\item Related work \url{http://koo.corpus.cam.ac.uk/drafts/tndjg-008-transactional-modelling-in-chisel.html}
%\item SV OOP is not available for synthesize, functional coverage, another test case could be my S4NOC, reference models are usually written in SystemC to avoid licenses cost for the SW developer
%\end{itemize}

\paragraph{Digital Design.}
VHDL and Verilog are the classic hardware description languages, first appeared in the 1980s.
SystemVerilog~\cite{SystemVerilog}, as an extension to Verilog, adds features from VHDL
for the hardware description and object-oriented features for verification.
Recent advances with SystemVerilog and Chisel \cite{chisel:dac2012, chisel:book} have
brought object-oriented programming into the digital design and verification process.

Chisel is a ``Hardware Construction Language'' embedded in Scala, to describe digital circuits~\cite{chisel:dac2012}.
Scala/Chisel brings object-oriented and functional programming into the world of digital design.
For hardware generation and testing, the full Scala language and Scala and Java libraries are available.
As Scala and Java's full power is available to the verification engineer,
the verification process is also made more efficient.


%Chisel is a hardware construction language embedded in Scala.
%Chisel allows the user to write hardware generators in Scala, an object-oriented and functional language.
%For example, we read in the string based schedules for a network-on-chip
%%\footnote{available at: \url{https://github.com/t-crest/s4noc/tree/master/noc/vhdl/generated}}
%and convert them with a few lines of Scala code into a hardware table to
%drive the multiplexer of the router and the network interface.
%
%Chisel is solely a hardware \emph{construction} language, and thus all valid Chisel code
%maps to synthesizable hardware.
%By separating the hardware construction and hardware verification languages,
%it becomes impossible to write non-synthesizable hardware and in turn, speeds up the design process.



%On open source collection of SystemVerilog classes build the Universal Verification Method,
%which is becoming popular in industry.
SystemVerilog has become a complex language with more than 250 keywords, and it is unclear
which tools support which language constructs for hardware description.
In contrast with Chisel, when the program compiles, it is synthesizable hardware.
Chisel is a small language, where the cheat sheet fits on two pages.
The power of Chisel comes from the embedding in Scala.
Furthermore, as classic hardware description languages are niche products, not
many tools or libraries are available. 
With Chisel on Scala we have the choice of different integrated development environments (IDE),
testing infrastructure (e.g., ScalaTest), and many free libraries.



%Chisel and Scala are executing on the Java virtual machine and therefore have a very good
%interoperability with Java. Therefore, we can leverage a large pool of Java libraries for
%hardware design and  verification.
%Furthermore, the name space of packets in Scala/Java simplifies integration of
%external components.
%Open source hardware components in Chisel can be organized like software
%libraries at Maven servers.






%%%ZF
\paragraph{Testing Methods.}
Testing is crucial for making software and hardware reliable. The higher
expectation of software quality and shrinking development cycle have
driven programming language researchers and software engineers to
develop a spectrum of \emph{automated testing} techniques.
For example, continuous integration~\cite{duvall2007continuous} 
implements an agile method that automatically runs manually written
%State-of-the-art research in software engineering
%and programming languages address the following two challenges to
%achieve full test automation.

SystemVerilog adds object-oriented concepts for the non-synthesizable verification code.
The SystemVerilog direct programming interface~\cite{Doulos:SV:dpi} allows the programmer to call
C functions inside a SystemVerilog (UVM) testbench.
This enables co-simulation with a ``golden model'' written in C, and the
testbench verifying the device under test (DUT).
With ChiselTest we can co-simulate with Java and Scala models and use the Java Native Interface
to co-simulate with models in C.
%
%The Java JNI (Java Native Interface) allows for a similar functionality in Java programs,
%allowing them to call C functions and use their functionality.
%By using Scala, which is built on Java, it is our hope to use the JNI together with Scala's test frameworks.
%The aim is to develop a framework for co-simulation with Scala/Chisel testers and a
%C-based golden model. This should allow companies to keep their existing C models,
%but move their simulation workflow into Scala/Chisel testers.

The digital design described in Chisel can be tested and verified with
ChiselTest~\cite{chisel:tester2}, a non-synthesizable testing framework for Chisel.
ChiselTest emphasizes usability and simplicity while providing ways to scale up complexity.
Fundamentally, ChiselTest is a Scala library that provides access into the simulator through
operations like poke (write value into circuit), peek (read value from circuit, into the test framework), and step (advance time).
As such, tests written in ChiselTest are just Scala programs, imperative code that runs one line after the next.
This structure uses the latest programming language developments that have been implemented into Scala
and provides a clean and concise interface, unlike approaches that attempt to reinvent the wheel like UVM.

Furthermore, ChiselTest tries to enable testing best practices from software engineering.
Its lightweight syntax encourages writing targeted unit tests by making small tests easy.
A clear and clean test code also enables the test-as-documentation pattern,
demonstrating a module's behavior from a temporal perspective.

\paragraph{Determining Specification.}
The first challenge is how to determine the expected code outcome, or
called \emph{specification}.  Software developers often document
specifications within the software itself.  \emph{Property-based testing}
provides a convenient way of formulating specification as constraints
expressed in a domain-specific
language~\cite{DBLP:conf/icfp/ClaessenH00}.  For example, developers
in the Scala language can specify ``$n^2>0$ for all integer $n$'' in the
property-based testing library ScalaCheck~\cite{nilsson2014scalacheck}
as follows:
\begin{lstlisting}[numbers=none]
val propSquare = forAll {(n: Int) => n * n > 0}
\end{lstlisting}
An efficient approach to obtaining such specifications is to learn from
historical bug patterns. Such patterns are gathered in pattern-based
software analysis tools, e.g., in
FindBug~\cite{DBLP:conf/paste/AyewahPMPZ07} or
Scalafix~\cite{web:scalafix}.  Recently, a specification for robotic
software is obtained by studying bugs fixed on the Robot
Operating System~\cite{nielsenFSW2020dependencybugs}.





%% ZF. This is commeented out because we have decided to not write on
%% the WP related to sanitizers.

%% Another way to generate specification is to use \emph{sanitizers}
%% in compilers. Sanitizers are a common compiler feature that
%% automatically injects run-time checks to capture certain failure
%% conditions~\cite{DBLP:conf/usenix/SerebryanyBPV12}. For example,
%% Google's undefined behavior sanitizer UBSAN checks whether a
%% floating-point variable equals 0 before a division by it; Google's
%% address sanitizer ASAN targets illegal memory access. While
%% sanitizers do not deal with functional correctness as unit tests
%% do, they check low-level, sometimes security-related properties,
%% such as use-after-free, array-index-out-of-bound, or race
%% conditions.

\paragraph{Generating Test Inputs.}
The second challenge is how to generate interesting test inputs.  The
state-of-the-practice uses \emph{fuzzing}, which has emerged as one of
the most effective testing techniques for discovering reliability
issues in software~\cite{takanen2018fuzzing}.  Google's OSS-Fuzz
project, for example, has filed over 20 000 bugs in 300 open-source
projects (as of June 2020)~\cite{web:oss-fuzz}.  Fuzzing techniques
generate random inputs and improve them based on observed code status,
e.g., crashes or code
coverage~\cite{DBLP:journals/tse/BohmePR19,DBLP:conf/pldi/FuS17}.
Property-based testing~\cite{DBLP:conf/icfp/ClaessenH00}, which is
initially designed for testing functional programming languages, uses
random data generation, a form of plain fuzzing. For example, by
invoking {\tt propSquare.check} with ScalaCheck (mentioned above), where {\tt
  propSquare} is the constraint $n^2>0$ for all integers $n$, we can
immediately get an input $n$ that falsifies the constraint:

\begin{lstlisting}[numbers=none]
scala> propSquare.check
       ! Falsified after 1 passed tests.
      > ARG_0: 0
      > ARG_0_ORIGINAL: 1083860448
\end{lstlisting}
 ScalaCheck first finds that n = 1083860448 produces an overflow
 wrapped to a negative (thus $n^2>0$ fails). The input is then
 ``shrunk'', in the terminology of property-based testing, to a smaller
 one, namely 0 in this case.



\section{Value Creation}

\todo{This is a weak section. Maybe we shall integrate the industry cooperation here.}

Digital technologies are currently the main driver for growth in
Denmark. Furthermore, the Covid-19 induced working from home, which gives a
further boost to digitalization right now.  With the
cooperation between researchers as DTU and ITU and Danish companies, this project
will allow Denmark to take the lead in digital research and
development.

Our value creation will targets the three following aspects.


\paragraph{Methodological Values. }
DTU and ITU will advance the research in the design and testing of
digital systems.  Our proposed approach provides a general software
engineering procedure that we plan to validate in the Robot Operating
System (closely related to co-PI's previous H2020 project on robotics) before
applying to real-world hardware systems.


\paragraph{Industrial Values.}
The industrial project partners will learn and get exposed to modern tools and methods to facilitate the designing and testing of digital systems dramatically. Our partner companies will benefit from our advanced software tool that generates test stimuli achieving high coverage of the hardware in simulation.

\paragraph{Educational Values.}
Two PhD students and additional master students will become highly educated engineers in need in the Danish or international digital systems design industry. Those students will also learn how to program the accelerators in the cloud for future applications. This research will drive the adaption of the education curriculum towards modern tools and agile methods.


\section{Research Plan}


%%%ZF
%\subsection{WP1}
As prerequisites, all researchers involved in this project will need to learn
languages and tools involved in the project and related work.
They will learn about the Scala programming language and ScalaCheck
(a Scala implementation of the property-based testing) on the software side.
On the hardware side,
the researchers need to get familiar with Chisel, SystemVerilog, and
UVM.


\paragraph{Hardware Generators.}

The productivity of hardware design can be greatly increased by developing so-called
hardware generators. A hardware generator is a program that can generate a configurable
hardware description.
Scala with functional programming is an excellent basis for developing a methodology for the
development of such hardware generators.


%\subsection{WP2}
\paragraph{Property-based Testing.}
As a starting point, we will use ScalaCheck, an implementation of
property-based testing in Scala, for hardware testing. We will invite
hardware developers to write properties as constraints (like
propSquare above), and then we will use ScalaCheck to validate or
refute those constraints. Property-based testing has seen notable
successes previously, such as in locating a long-standing concurrency
bug in the Erlang database server~\cite{DBLP:conf/erlang/HughesB11};
it was also used by Ericsson to test its media
proxy~\cite{DBLP:conf/erlang/ArtsHJW06}, by Volvo to test car
communications protocols~\cite{DBLP:conf/icst/ArtsHNS15}.


%\paragraph{Co-simulation.}
%
%\todo{A figure would be nice.}
%
%\begin{itemize}
%\item One use case for evaluation: cosimulation of a RISC-V simulator (Tommy) with an OS RISC-V HW 
%\item Ptolemy~\cite{ptolemyII-book} can be used to co-simulate the environment, supporting a model based design.
%\item Integration of C/C++ based models in the verification with Scala
%\item Java/Scala in UVM
%\item Hw/sw co-verification with Scala and so on, e.g., run an application on a SW processor model exploring some hardware artifacts (could be S4NOC)
%\item Model based design (Jan) with co-simulation
%\end{itemize}
%
%\paragraph{Assertions.}
%
%\begin{itemize}
%\item assertions during simulation
%\item Assertion ave been long part in SW, begin of C, but seldom used in HW and more complex assertions are interesting, such as when req is asserted, an ack has to become active within 5 clock cycles
%\end{itemize}

\paragraph{Constraint Random Testing and Fuzzing Techniques.}

Constraint random testing in hardware is similar to fuzzing in software testing.
We will add support for constrain random testing to ChiselTest~\cite{chisel:tester2}.
Once we have tools to instrument the hardware under test with a user-written
specification or automatically generated specification, we will be able to use fuzzing techniques
to validate those specifications. To this end, we will work
with our industrial collaborators to get access use cases serving
as the DUT.

\paragraph{Test Coverage.}

Code coverage is a useful tool for verifying digital designs
since it allows one to see which parts of their design have actually been tested correctly. 
We will implement coverage inside of the execution engine of the Chisel simulator using a technique
presented by Ira. D. Baxter~\cite{branch-cov-made-easy:2002}.
We will add a method to specify \textit{functional coverage points}, also known as
\textit{coverage groups} in SystemVerilog.


%Treadle, which is a simulation engine for Chisel, does not contain support for measuring code coverage.
%We will added branch coverage to Treadle so that one can see which lines of
%LoFIRRTL code, the intermediate representation used inside of Treadle, were covered by a series of tests
%and then know, using that information, which multiplexer paths were tested. 
%Furthermore, we plan to map the results obtained with LoFIRRTL code back to the source Chisel description.
%This mapping can be done using treadle's internal ``source trackers'' that associate some FIRRTL lines back
%to their chisel source. Once that is done, it will be interesting to move from branch coverage to functional
%coverage, which would require a way to define \textit{functional coverage points} also known as
%\textit{coverage groups} in SystemVerilog.

%\begin{itemize}
%\item On coverage and cover points in proposal (coverage of the RTL hardware, but also on the generator (Jack's comment, see \url{https://gitter.im/freechipsproject/chisel3?at=5f63c878603d0b37f43b67f3})
%\item Also have range coverage on individual signals and the matrix of 2 or more (see UVM example).
%\end{itemize}
%


\paragraph{Verification Framework.}

We will develop an object-oriented and functional framework for verification in Scala.
This framework is inspired by UVM, but will leverage Scala's conciseness with the
combination of object-oriented programming with functional programming.
%An initial experiment of testing the accumulator circuit of the Leros processor~\cite{leros:arcs2019}
%showed the that a test written with UVM was about 800 lines of code, where a Scala based
%test was around 80 lines of code~\cite{verify:chisel:2020}.
%However, UVM supports more functionality that a plain ChiselTest in Scala.
%
Within our verification framework, we will support mixed language verification.
Verilog can easily be combined with Chisel, as Chisel generates Verilog, and
we will use ChiselTest as a driver for the open-source Verilog simulator Verlator.
With Yosys synthesis suite~\cite{Yosys} and GHDL~\cite{ghdl}
we will translate VHDL into Verilog.

%A verification method is only usable when it can handle mixed-source designs.
%This means a Scala driven method must be able to test components written in Verilog,
%VHDL, and SystemVerilog.
%
%Chisel has support for black boxes, which allows the use of Verilog code within the Chisel design.
%Therefore, it is relatively easy to integrate Verilog components when wrapped into a black box.
%However, this forces Chisel to use Verilator instead of Treadle to run the simulation, impacting
%startup time.
%
%Chisel does not fully support VHDL. It can support VHDL using VCS, but there is no
%open-source solution available for VHDL simulation. For companies with a lot of source code written in VHDL this is a concern, as they must be able to integrate their existing IP in a Scala/Chisel based design and verification workflow.
%All major commercial simulation and synthesis tools support mixed-language designs, but no open-source tools exist that provide the same functionality.
%
%To alleviate this issue, the open-source Yosys synthesis suite \cite{Yosys} can be used. Yosys is an open-source digital hardware synthesis suite for Verilog. Yosys also has a variety of plugins, one of these being a plugin for using GHDL \cite{ghdl}, an open-source VHDL simulator. By using Yosys in conjunction with GHDL, VHDL files are compiled to an RTL-based intermediate representation, which is then written to a Verilog file using Yosys. GHDL has full support for IEEE 1076 VHDL 1987, 1993, 2002, and a subset of 2008. The workflow can be seen in Figure \ref{fig:VHDL2Verilog}. A working solution named VHDL2Verilog has been made for this, which has been tested with certain simple VHDL designs \cite{vhdl2verilog}.



%\begin{itemize}
%\item Bus functional models
%\item Multiple languages
%\item Still talk about a small example taking it through all variations
%\item WP on VHDL generation from Chisel for better verification
%\item p69: HDL models are SW projects
%\end{itemize}



%% Let us consider this one below as a bonus WP in our backyard.
%% \subsection{WP2} 
%% We can then consider to generate specification automatically, \`a la
%% sanitizers used in the compiler techniques. This WP focuses on
%% low-level properties derived from the program syntax, {\em, e.g.},
%% accessing an array should be within a bound, or performing an
%% arithmetic calculation should not overflow.



%% \paragraph{MS2.} The milestone in this WP is to develop a program transformer
%% that injects in the software source, or its binary form, a list of
%% low-level specifications.

%% Related to this WP, our project collaborator Zhoulai Fu (together with
%% a hired security hacker) used sanitizers to find more than 200 bugs in
%% the Robot Operating System
%% (ROS)~\cite{web:ros-sanitizer-logs}. Besides, Zhoulai Fu has developed
%% several program transformers previously for testing floating-point
%% computation~\cite{DBLP:conf/pldi/FuS19,DBLP:conf/oopsla/FuBS15}.


\todo{Co-verification}

\paragraph{Learning Specification from Historical Bug Patterns.}
We will consider generating specifications from previously known bug
patterns.  We will collaborate with our hardware engineers to explore
a history of hardware issues triggered by software defects and get
patterns from which we generate specification.  Such bug patterns in
Java, for example, can be found in \cite{web:findbugs_bugs}.  This
step will generate specification in a syntax-driven way, relating to
the expected functional behavior of the hardware.



%% Related to this, our collaborator Zhoulai Fu has worked on a
%% comprehensive study of developing bug finders by learning from history
%% bug patterns in the Robot Operating
%% System~\cite{nielsenFSW2020dependencybugs} 


%% Let us consider this as another backyard WP.
%% \subsection{WP5} We plan to deploy our automated testing solution  in a continuous
%% integration service, presumably on the cloud, such as Jenkins or
%% Travis~\cite{DBLP:journals/tse/GallabaM20}. For each code commit, the
%% service will automatically generate the specification as implemented
%% in WP2 and WP3, and then will generate test data with a fuzzing engine
%% as in WP4.

%% \paragraph{MS5.} The milestone is to have the server established that implements
%% the workflow. It should produce a detailed status report for each
%% commit.  A similar kind of status report can be found in Google's
%% OSS-Fuzz (Continuous Fuzzing for Open Source
%% Software)~\cite{web:oss-fuzz} for example.


\paragraph*{Dissemination and Publication}

Scientific results will be published and presented at international
conferences (e.g., DATE, DAC, CAV, FPL, ISCAS, FPGA) and in relevant scientific journals
and two PhD theses.
%We expect that most tasks will result in at least one publication.
%One PhD theses will publish the results from the project.
We aim to publish in open access, to a large extent, in the gold open access model.
However, publishers such as ACM also allow publishing in green open access
at no additional cost, where a pre-print version of a paper can be uploaded,
for example, to ArXiv.

The results from the project will be available as open-source under the
industry-friendly BSD license.
%Open-source research projects attract
%other researchers, developers, and industrial partners
%to use and build on the results of the project.
A project web site will host the project documentation, the published papers, and the design's source code.
We will provide unrestricted and cost-free digital access to all research and development results.
%
We will use the developed method and tools to train a new generation of HW/SW engineers
at DTU and ITU.
In the middle of the project, we will have a coordination workshop with the project partners.
In the end, we will organize a design and verification workshop, including a hands-on tutorial,
open to all interested companies, and students from DTU and ITU.

\section{Practical Feasibility}

The Embedded Systems Engineering section at DTU Compute provides
the intellectual environment and the infrastructure (e.g., regression test server...) that we need for an ambitious research project.
Furthermore, DTU Compute provides the infrastructure (e.g., an automatic test
environment for regressions tests, web server).
\todo{Add ITU}

%\paragraph*{Internationalization}
%
%\todo{Still waiting from SiFive for an ok for a stay.}



\paragraph*{Industrial Cooperation}

Danish industry in digital design for ASICs and FPGAs is currently in transition from using traditional
test benches written in VHDL and Verilog to a verification method based on constraint random
test generation with tools such as UVM. Therefore, the HaaS project is just-in-time to support this
transition.

The topic is relevant to several companies in Denmark. Following companies provided a commitment: Napatech (Jesper Birch), GN Hearing (Mark Brooks), Comcores (Thomas Noergaard, Joergen Carstensen), Teledyne Reson (Morten Rytter, Simon Andersen), WSAudiology (Ketil Julsgaard), and Microchip (Thomas Aakjer)


Following companies are considered additional partners: SyoSil (Jacob Sander Andersen), Synopsis (Martine Chegaray), and Demant (Anders Hebsgaard).
\todo{Missing replies.}

\todo{More concrete text on contribution and cooperation. Probably one short abstract per company}

Following {\bf six companies are interested in the HaaS project} and support it with use cases.

Napatech...

GN Hearing...

Comcores is interested in improving the throughput with HaaS.

Teledyne is interested in exploring Chisel for development and testing with the compatibility of VHDL and Verilog.

WSAudiology provides a decimation filter use case for our verification framework.

Microchip is interested in using Chisel to improve the design throughput and
provides a hardware use case to sort Ethernet packets according to their deadlines
for a time-sensitive network switch.

% Synopsys provides tool support for the project.
%Napatech will follow the project and offer challenges to the project.
%Syosil will provide consulting and training in connection with any questions related to functional verification and UVM.


\paragraph*{Use Cases}

We will have several non-trivial use cases from industry and from our own development to verify
our development.
Microchip provides the specification of a hardware sorting algorithm.
From WSA we will use a decimation filter written in VHDL.
From our research we will use a multicore device, a network-on-chip~\cite{s4noc:nocarc2019},
to explore concurrent, transaction based verification.

\paragraph*{Human Resources}

For the HaaS project, we request funding of two PhD students and a student researcher.
Each of the senior researchers will contribute to the HaaS research project.
%
%We intend to build a group with one PhD student, one postdoc, and
%two senior researchers at DTU.
%
%Quoted from the Diversity and Gender statement at DTU:
%``Diversity, equal treatment, and equality are integral to DTU, being an international
%university in scope and standard, and are fundamental principles underlying DTU's
%expectations of respect and equality''.
As the already named researchers are all male, we will actively search
for female researchers for the PhD positions.
%However, the PhD position will be announced openly and men and women
%will have equal opportunities for applying.


{\bf Martin Schoeberl (MS)} is associate professor at DTU Compute and is the PI.
%His research interest is in computer architecture for real-time systems.
During his stay
at UCB in 2012 he picked up Chisel and brought it to DTU in research and teaching.
Martin has {\bf written the Chisel textbook}~\cite{chisel:book}, which has been translated
into Chinese, Japanese, and Vietnamise. Martin is a {\bf member of the Technical Advisory Committee for
Chisel} and therefore keeps the work of HaaS in sync with the Chisel main development.
Martin is part of the regular Chisel developer meeting and a meeting on verification with
UC Berkeley researchers and developers from SiFive, a startup in silicon valley.


{\bf Jan Madsen (JM)} is professor and deputy director of DTU Compute.
His research spans methods
and tools for systems engineering of computing systems, embedded systems-on-a-chip,
Cyber-Physical Systems (Internet-of-Things), microfluidic biochips (Lab-on-Chip), and
synthetic biology (molecular computing). 

\todo{Can we have a student researcher in the budget? Question to Ellen/Jan.}
{\bf Kasper Hesse (KH)} is student at DTU with a strong background in digital design.
He has already some knowledge in UVM and can therefore help the project to be productive
from day one.

{\bf Zhoulai Fu (ZF)} is an assistant professor at ITU. His research
spans programming language theory and software engineering
techniques. Related to this project, he worked on developing automated
testing techniques in achieving high code coverage in scientific
programs and detecting bugs in robotic software.

{\bf Peter Sestoft (PS)} is professor and head of department at the IT University of Copenhagen.
His research focus is programming languages, and especially functional, parallel, domain-specific
and declarative languages, their description, formal modeling, analysis, transformation, and implementation.
Most of his books have appeared with leading international academic publishers.

{\bf PhD1/2} we are looking for two PhD candidates with a background in computer engineering
with interest in digital design and programming languages.

%\subsection*{Experimental Facilities}
%
%Development and simulation of the RTAI hardware can be
%performed on standard desktop PCs.
%For evaluation of individual design components small and cheap FPGA
%boards, that are already available, can be used. For the evaluation of the
%full system design of a switches with ANNs, we intend to buy three high-performance
%FPGA boards.
%%
%The needed software (e.g., VHDL simulation, FPGA compilation) is freely available.

\todo{Co-verification}

%\vspace{-2mm}
\begin{table*}% [h!]
{\small
  \begin{center}
    \begin{tabular}{lccp{110mm}l}
      \toprule
      Task                   & PM  & Person    & Description                                                                                   \\
      \midrule
      Recruiting             & 1   & MS and PS & Recruiting of the two PhDs.                                                                    \\
      \midrule
      UVM                    & 3   & KH        & Develop a UVM verification example on a complex multicore device.          \\
      Prepare1                & 3   & PhD1      & Exploring related work and learning Chisel, SystemVerilog, and UVM.                           \\
      Cover                  & 6   & PhD1      & Adding constraint random testing and coverage to ChiselTest.                     \\
      Generate               & 6   & PhD1      & Developing a hardware generator methodology by using functional programming in Scala.         \\
      Framework              & 9   & PhD1      & Development of an object-oriented and functional testing framework.                      \\
      \midrule
      Prepare2                    & 3   & PhD2      & Explore related work on property-based testing and ScalaCheck.                   \\
      Spec          & 6   & PhD2      & Collaborate with hardware developers to manually inject both valid and faulty specifications. \\
      Testing & 6   & PhD2      & Develop property-based testing the specifications with ScalaCheck.                                                      \\
      Bugbase           & 3   & PhD2      & Create a database from historical bug patterns from hardware development.                   \\
      Implement              & 6   & PhD2      & Implement a program transformer to convert the bug patterns into %
  additional, meaningful  specification in code (\`a la FindBugs or ScalaFix).                                                                 \\
%      Implement              & 6   & PhD2      & Implement a user-friendly tool for automating the testing process.                            \\
      \midrule
      Explore                & 2x4 & PhD1/2    & Detect and report bugs with the implemented tool and incorporate their feedback into a detailed
      empirical bug study.                                                                                                                      \\
      Eval                   & 2x4 & PhD1/2    & Evaluation of the results collected with the industry use cases.                                                     \\
      Thesis                 & 4   & PhD1/2    & Thesis writing und submission.                                                                \\
      \midrule
      TeachSW                 &  2  & JM            & Make Chisel and its design methods accessible for software designers. \\
      Chapter                & 2   & MS        & Adding a chapter on verification to the Chisel book~\cite{chisel:book}.                       \\
      \bottomrule
    \end{tabular}
  \end{center}
  \caption{Tasks for HaaS}\label{tab:packages}
}
\end{table*}





%\vspace{-5mm}
\paragraph*{Tasks, Milestones, and Timetable}


%\paragraph{MS1.} The milestone is to have a empirical study on the effectiveness of our approach. In particular,  we will consider testing  two kinds of properties. The first kind is injected faults, of which we will measure our capability in detecting the faults. Another category is the expected code behavior, on which our approach should  pass without raising  warnings. 
%
%
%\paragraph{MS2.} We will create  a database from historical  bug patterns.
%A program transformer will be implemented to convert the bug patterns
%into code specification. The generated specification for our benchmark
%should be validated by our hardware developers, with which a
%refinement step can be introduced to improve the database.
%
%\paragraph{MS3.} The milestone  is to have a list of bugs reported, and a detailed case study (whether the bugs are
%trivial, false alarms, for example) Besides, coverage statistics will
%be collected to measure comprehensiveness of our tests.


The project is divided into several tasks.
%A researcher will be
%assigned to each task as its main developer.
%A cooperative
%working style will be encouraged so, that the experience and
%knowledge of the different team members will optimally be utilized.
For each task, the time is given in person-months (PM). The Gantt chart shows
the project schedule.
For an assessment of the project's success, we plan the following milestones:

\textbf{M1} (Month 6): The PhD students are selected and employed.

\textbf{M2} (Month 15): Tools have been learned, and first artifacts have been developed.

\textbf{M3} (Month 30): All development has been finished, and the different components
can be used for exploration and evaluation of the results with the industrial use cases.

\textbf{M4} (Month 48): The project has finished, and two PhD theses have been handed in.

%\vspace{-5mm}
\begin{figure*}
\centering
\begin{ganttchart}[vgrid,hgrid,bar/.style={fill=gray},
x unit=3mm, % horizontal squeezing
%y unit chart=5mm, % vertical squeezing
y unit title=8mm,y unit chart=4.5mm, milestone top shift=.15, milestone height=.2mm, % very tight format
title label font=\footnotesize,
bar label font=\footnotesize,
milestone label font=\footnotesize,
]{1}{48}
% labels
%\gantttitle{\textbf{\normalsize{RTAI Gantt chart}}}{42} \\
\gantttitlelist{1,...,48}{1} \\
% tasks, groups and milestones
\ganttbar[name=r1]{Recruiting}{1}{6} \\
\ganttmilestone[name=m1]{Milestone 1}{6} \\
\ganttbar[name=t1]{UVM}{1}{9} \\
\ganttbar[name=t2]{Prepare1}{7}{9} \\
\ganttbar[name=t3]{Cover}{10}{15} \\
\ganttmilestone[name=m2]{Milestone 2}{15} \\
\ganttbar[name=t4]{Generate}{16}{21} \\
\ganttbar[name=t41]{Framework}{22}{30} \\
\ganttbar[name=t5]{Perpare2}{7}{9} \\
\ganttbar[name=t6]{Spec}{10}{15} \\
\ganttbar[name=t7]{Testing}{16}{21} \\
\ganttbar[name=t8]{Bugbase}{22}{24} \\
\ganttbar[name=t9]{Implement}{25}{30} \\
\ganttmilestone[name=m3]{Milestone 3}{30} \\
\ganttbar[name=t10]{Explore}{31}{34} \\
\ganttbar[name=t11]{Eval}{35}{39} \\
\ganttbar[name=t12]{Thesis}{40}{48} \\
\ganttbar[name=t13]{TeachSW}{40}{46} \\
\ganttbar[name=t14]{Chapter}{40}{46} \\
\ganttmilestone[name=m4]{Milestone 4}{48}

% relations
\ganttlink{r1}{m1}
\ganttlink{m1}{t2}
\ganttlink{t1}{m2}
\ganttlink{t2}{t3}
\ganttlink{t3}{m2}
\ganttlink{m2}{t4}
\ganttlink{t4}{t41}
\ganttlink{t41}{m3}

\ganttlink{m1}{t5}
\ganttlink{t5}{t6}
\ganttlink{t6}{m2}
\ganttlink{m2}{t7}
\ganttlink{t7}{t8}
\ganttlink{t8}{t9}
\ganttlink{t9}{m3}
\ganttlink{m3}{t10}
\ganttlink{t10}{t11}
\ganttlink{t11}{m4}
\ganttlink{t12}{m4}
\ganttlink{t13}{m4}

\end{ganttchart}

\caption{The Gantt chart of HaaS}\label{fig:gantt}
\end{figure*}





\newpage
\newpage
\small
\bibliographystyle{abbrv}
%\bibliography{myown,jsp,noc,misc,msbib}
\bibliography{../../msbib,testing,../../chisel-uvm}

\end{document}

\newpage

\section{Ideas, Questions, TODOs}



\begin{itemize}
\item Include proposal from Jesper Birch (in Word document)
\item Take material from workshop paper
\item Look at the TODO list in infinit project
\item Widex is interested in Chisel, may join a DFF proposal
\item Peter Sestoft + microsoft cloud into Chisel project
\item DeepSpec or DeepSp?? end-to-end for Chisel
\item Support letter from UCB
\item External stay at UCB
\item Richard should be part of it
\item One PhD at ITU and PhD or postdoc (Lefteris at DTU)
\item Kasper n hours per week plus maybe part time Torur
\item Thomas from Microsemi on board
\item Konstantin Vinogradov <const.vin@gmail.com> from Widex is interested in a (industrial) PhD, might be a named candidate
\item Teledyn is interested
\end{itemize}

\subsection{Contacts}

%Jesper Birch <jb@napatech.com> sent a word document and is interested.
%
%Teledyne:
%
%"Rytter, Morten (INT)" <Morten.Rytter@Teledyne.com>
%
%simon.andersen@teledyne.com
%


One option would be for us to deliver test cases and discussions of what is needed for our company. 



\subsection{TODO}

Send proposal draft proposal and of August to Jesper, Thomas, ...

\section{From InfinIT}

Digital systems are already an integral part of our life. These systems are built out of microprocessors, application specific integrated circuits (ASICs), and field-programmable gate arrays (FPGAs). Several companies in Denmark are building digital systems. To increase competitivity of those companies, we need tools and methods to increase the productivity in designing and especially testing digital systems. Compared to software development and testing, digital design and testing methods and tools lack several decades of development. Within this project we plan to leverage software development and testing methods for digital design. This project explores the hardware construction language Chisel with Scala and the Universal Verification Method (UVM) with SystemVerilog for design and test of digital systems.



UVM is becoming an industry standard for design verification. On the other hand there is an active development on a new hardware construction language, called Chisel. Chisel is embedded in Scala to write so-called hardware generators. Chisel is also called: software defined hardware. Another feature of the Chisel/Scala combination is to write models of the environment of the hardware design in Scala. As an example take a network interface (e.g., Ethernet) written as a high-level model in Scala connected to a microprocessor written in Chisel. With this example we are able to develop and test network code on the microprocessor, which is our digital design under test.



As a first step we will explore and compare the two approaches: UVM/SystemVerilog and Chisel/Scala. When we generate hardware from Chisel, we generate a Verilog description of the digital circuit. This Verilog description can further be tested within UVM (plain Verilog is valid SystemVerilog). As a next step, we will explore how test, simulation, and verification code written in Scala to develop the Chisel description of the digital circuit can be reused at the UVM level to test the generated Verilog description of the circuit.


Modern software techniques can be applied on testing where plausible. For example, fuzzing is a mature solution for producing random and yet meaningful inputs to trigger program failures. Symbolic execution explores program paths systematically via constraint solving.



We are in contact with the developers of Chisel at the University of California in Berkeley, and especially with Richard Lin, who is developing the new testing framework for Chisel. Richard is interested in this project and the integration with UVM. Therefore, we agreed to have a cooperation meeting during the project at UC Berkeley.



The project fits into the Infinit topic of IoT. The things of IoT are digital systems, often small and application specific systems. Application specific systems are built out of digital systems either with a dedicated ASIC or an FPGA.



This mini-project will be executed in close cooperation with Microchip, WSA, Synopsys, and Syosil. 



The students involved in the research project well then be well educated future engineers for digital system design and verification.



3. Aktiviteter (beskriv) 1. Learning and exploring SystemVerilog/UVM (with Synopsys)
2. Learning and exploring Chisel/Scala
3. Defining two use cases together with Microchip
4. Developing the two use-cases in Chisel and SystemVerilog with a comparison
5. Developing the verification environment including high-level models of the environment in UVM and Chisel/Scala with a comparison
6. Application of the UVM verification of the Chisel generated Verilog code
7. Scala based testing and verification on top of UVM
8. Develop an open course on verification of digital systems for DTU and use in industry




4. MilestonesKnowledge of the tools30/4/2020Definitions of the use cases31/5/2020Use cases developed30/7/2020Verification and high-level models developed31/8/2020Cross verification from Scala to UVM functional, course material finalize31/10/2020

5. Deltagere



DTU, CVR-nr. 30 06 09 46, Martin Schoeberl (project lead) (masca@dtu.dk) and Jan Madsen (jama@dtu.dk)
DTU will develop the use cases and the verification environment with UVM and Chisel/Scala. DTU will transfer knowledge on Chisel to Microchip and WSA.
ITU, CVR-nr. 29 05 77 53, Peter Sestoft (sestoft@itu.dk) and Zhoulai Fu (zhfu@itu.dk)
ITU will apply methods from software testing to digital hardware verification.
Aarhus Universitet, CVR-nr: 31119103, Farshad Moradi (moradi@eng.au.dk)
AU will explore UVM verification of Chisel generated Verilog code.
Microchip Semiconductor Corp. A/S, CVR-nr. 24224694, Thomas Aakjer (Thomas.Aakjer@microchip.com)
Microchip will provide use cases for the research in design and verification where DTU can explore Chisel with Scala.
WS Audiology Denmark A/S, CVR-nr. 40296638, Ketil Julsgaard (ketil.julsgaard@wsa.com)
WSA will provide digital-signal processing use cases for cosimulation of a Chisel description with a high-level description.
Synopsys, CVR-nr. 25600568, Martine Chegaray (Martine.Chegaray@synopsys.com)
Synopsis will provide the tools for UVM for the project and guide the usage.
Syosil Aps, CVR-nr. 29399417, Jacob Sander Andersen (jacob@syosil.com)
Syosil will support the researchers with education in using UVM.


6. Resultater og vision for 

The vision of the project is a highly productive method for designing and (more importantly) verification of digital systems by a combination of the modern hardware construction language Chisel/Scala with the industry standard UVM.
7. Videnspredning

The research work will be documented by publications and presented at relevant conferences (for example DATE), funded by other means, not by this project.

At the end of the project we will present the method at a workshop open for Danish industry in digital system design.

The new development and verification method will be used and taught in courses on digital electronics at DTU.

\end{document}
