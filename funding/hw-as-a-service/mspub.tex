\documentclass[%draft,
    a4paper,
    11pt, % use explicit paper size
    headinclude, footexclude,
    notitlepage,
    headsepline,
    pointlessnumbers,
    ]{scrartcl}
\usepackage{pslatex} % -- times instead of computer modern

\typearea{14}
\usepackage{scrpage2} % for headers
 \setkomafont{pagehead}{\scshape\small}
 \setkomafont{pagenumber}{\scshape\small}
 \ihead[]{Martin Schoeberl}
 \ohead[]{Publications}

% \ofoot[]{} \cfoot[]{} \ifoot[]{}

\usepackage{hyperref}
\usepackage{booktabs}
\usepackage{graphicx}
\usepackage{amsmath}
\usepackage{dcolumn}
\newcommand{\cc}[1]{\multicolumn{1}{c}{#1}}
\newcolumntype{d}[1]{D{.}{.}{#1}}
\usepackage{boxedminipage}

\newcommand{\excelwidth}{\columnwidth}

\newcommand{\code}[1]{{\textsf{#1}}}


\begin{document}

\pagestyle{scrheadings}

\begin{center}
\vspace{3cm}
{\usekomafont{title}\huge Publications Martin Schoeberl}\\
\bigskip
\bigskip
\end{center}

I have published four books, 27 journal articles, one book chapter, one patent, and 150
papers in peer reviewed conferences and workshops; 74 of the publications
as main author (38 of the 74 as single author).
According to Google Scholar,\footnote{Google profile at
\url{http://scholar.google.com/citations?user=wiRNmwUAAAAJ&hl=en&oi=ao}
}
my papers are cited 4176 times and my h index is 36.
% The first article (JSA 2008) was the \textbf{most cited article at JSA} of articles published in the last five years.\footnote{\url{http://www.elsevier.com/wps/find/journaldescription.cws_home/505616/description}}

\subsection*{Five Most Important Publications}

\begin{itemize}

\item Martin Schoeberl. A Java processor architecture for
embedded real-time systems. {\em Journal of Systems
Architecture}, 54/1--2:265--286, 2008.

This article summarizes my PhD work on a time-predictable Java processor
(called JOP). This article was the \textbf{most cited article at JSA} of articles
published in the last five years for several years (it is now out of the 5 year
window.) Its citation count is 248.

\item Martin Schoeberl, Wolfgang Puffitsch, Rasmus~Ulslev
    Pedersen, and Benedikt Huber. Worst-case execution time
  analysis for a {Java} processor. {\em Software: Practice and
  Experience}, 40/6:507--542, 2010.
  
This article presents the worst-cases execution time (WCET) analysis tool for
the Java processor JOP.
The hardware design of JOP together with this tool is currently the only
solution for embedded Java where the worst-case execution time can
be predicted statically.

\item Martin Schoeberl. Time-predictable computer architecture.
    {\em EURASIP Journal on Embedded Systems}, vol. 2009, Article
    ID 758480:17 pages, 2009.
    
This article take the basic ideas from the JOP design on time-predictable
computer architecture towards general purpose processors. This article was
the main article for my habilitation thesis, where I {\bf establish time-predictable
computer architecture as a research direction}.
Furthermore, the T-CREST project was based on the ideas presented in this article.
   
%\item Martin Schoeberl. A time predictable instruction cache for
%    a {J}ava processor. In {\em On the Move to Meaningful
%    Internet Systems 2004: Workshop on {J}ava Technologies for
%  Real-Time and Embedded Systems (JTRES 2004)}, volume 3292 of
%  {\em LNCS}, pages 371--382, Agia Napa, Cyprus, October 2004.
%  Springer.
%
%This paper presents one of my first time-predictable features for a processor,
%the so called method cache, to simplify WCET analysis. This idea has been further
%explored by a whole PhD thesis by Stefan Metzlaff in Augsburg.

\item Christof Pitter and Martin Schoeberl.
A real-time {Java} chip-multiprocessor.
{\em ACM Trans. Embed. Comput. Syst.}, 10(1):9:1--34, 2010.

Together with my first PhD student Christof Pitter we built a time-predictable
chip-multiprocessor based on time-predictable memory arbitration and
several Java processors. To the best of my knowledge this was the first multicore
processor that was WCET analyzable and supported by a WCET analysis
tool.

\item Martin Schoeberl, Sahar Abbaspour, Benny Akesson, Neil Audsley, Raffaele Capasso, Jamie Garside, Kees Goossens, Sven Goossens, Scott Hansen, Reinhold Heckmann, Stefan Hepp, Benedikt Huber, Alexander Jordan, Evangelia Kasapaki, Jens Knoop, Yonghui Li, Daniel Prokesch, Wolfgang Puffitsch, Peter Puschner, Andr\'{e} Rocha, Cl\'{a}udio Silva, Jens Spars{\o}, and Alessandro Tocchi.
 T-CREST: Time-predictable Multi-Core Architecture for Embedded Systems.
 \emph{Journal of Systems Architecture} 61(9):449--471, 2015.
 
 
This article summarizes the research and development work within the EC funded
project T-CREST.
 
\end{itemize}

\subsection*{Books}

\begin{enumerate}

\item Martin Schoeberl. {\em Digital Design with Chisel}.
Number ISBN 9781689336031. Kindle Direct Publishing, August 2019. 

\item Martin Schoeberl. {\em JOP Reference Handbook: Building
    Embedded Systems with a Java Processor}. Number ISBN
978-1438239699. CreateSpace, August 2009.

\item Martin Schoeberl. {\em JOP: A Java Optimized Processor for
    Embedded Real-Time Systems}. Number ISBN 978-3-8364-8086-4.
    VDM Verlag Dr. M{\"u}ller, July 2008.

\item Doug Locke, B.~Scott Andersen, Ben Brosgol, Mike Fulton, Thomas Henties,
  James~J. Hunt, Johan~Olm\"{u}tz Nielsen, Kelvin Nilsen, Martin Schoeberl,
  Joyce Tokar, Jan Vitek, and Andy Wellings.
 Safety-critical {Java} technology specification, public draft, 2011.

\end{enumerate}

\subsection*{Book Chapter}

\begin{enumerate}

\item Martin Schoeberl.
 Hardware support for embedded {Java}.
 In M.~Teresa Higuera-Toledano and Andy~J. Wellings, editors, {\em
  Distributed, Embedded and Real-time {Java} Systems}, pages 159--176. Springer
  US, 2012.
\end{enumerate}

\subsection*{Patent}

\begin{enumerate}

\item Martin Schoeberl. Instruction Cache f\"ur Echtzeitsysteme,
    April 2006. Austrian patent AT 500.858.
\end{enumerate}


\subsection*{Theses}

\begin{enumerate}

\item Martin Schoeberl.
 Time-predictable computer architecture.
 Habilitation thesis, Institut for Computer Engineering, Vienna
  University of Technology, September 2009.

\item Martin Schoeberl. {\em JOP: A Java Optimized Processor for
    Embedded Real-Time Systems}. PhD thesis, Vienna University of
    Technology, 2005.
\end{enumerate}


\subsection*{Journal Articles}

\begin{enumerate}

\subsubsection*{2020}

\item Eleftherios Kyriakakis, Maja Lund, Luca Pezzarossa, Jens Spars{\o}, and {\bf Martin Schoeberl }.
 A time-predictable open-source TTEthernet end-system.
 \emph{Journal of Systems Architecture} 108:101744, 2020.


\subsubsection*{2019}

\item Morten B. Petersen, Anthon V. Riber, Simon T. Andersen, and {\bf Martin Schoeberl }.
 Time-predictable Distributed Shared On-Chip Memory.
 \emph{Microprocessors and Microsystems} 2019.

\item T{\'o}rur Biskopst{\o} Str{\o}m, Jens Spars{\o}, and {\bf Martin Schoeberl }.
 Hardlock: Real-time multicore locking.
 \emph{Journal of Systems Architecture} 97:467--476, 2019.

\item Emad Jacob Maroun, Henrik Enggaard Hansen, Andreas Toftegaard Kristensen, and {\bf Martin Schoeberl }.
 Time-predictable synchronization support with a shared scratchpad memory.
 \emph{Microprocessors and Microsystems} 64:34--42, 2019.


\subsubsection*{2018}

\item {\bf Martin Schoeberl }, Wolfgang Puffitsch, Stefan Hepp, Benedikt Huber, and Daniel Prokesch.
 Patmos: A Time-predictable Microprocessor.
 \emph{Real-Time Systems} 54(2):389--423, 2018.

\item Luca Pezzarossa, Andreas Toftegaard Kristensen, {\bf Martin Schoeberl }, and Jens Spars{\o}.
 Using Dynamic Partial Reconfiguration of FPGAs in Real-Time Systems.
 \emph{Microprocessors and Microsystems} 61:198--206, 2018.

\item {\bf Martin Schoeberl }, Luca Pezzarossa, and Jens Spars{\o}.
 A Multicore Processor for Time-Critical Applications.
 \emph{IEEE Design Test} 35:38--47, 2018.

\item Rasmus Ulslev Pedersen and {\bf Martin Schoeberl }.
 Direct garbage collection: two-fold speedup for managed language embedded systems.
 \emph{International Journal of Embedded Systems} 10(5):394--405, 2018.


\subsubsection*{2017}

\item {\bf Martin Schoeberl }, Andreas Engelbredt Dalsgaard, Rene Rydhof Hansen, Stephan E. Korsholm, Anders P. Ravn, Juan Ricardo Rios Rivas, Torur Biskopst{\o} Str{\o}m, Hans S{\o}ndergaard, Andy Wellings, and Shuai Zhao.
 Safety-critical Java for embedded systems.
 \emph{Concurrency and Computation: Practice and Experience} 29(22), 2017.

\item T{\'o}rur Biskopst{\o} Str{\o}m, Wolfgang Puffitsch, and {\bf Martin Schoeberl }.
 Hardware Locks for a Real-Time Java Chip-Multiprocessor.
 \emph{Concurrency and Computation: Practice and Experience} 29(6):e3950--n/a, 2017.

\item Rasmus Bo S{\o}rensen, Luca Pezzarossa, {\bf Martin Schoeberl }, and Jens Spars{\o}.
 A resource-efficient network interface supporting low latency reconfiguration of virtual circuits in time-division multiplexing networks-on-chip.
 \emph{Journal of Systems Architecture} 74(Supplement C):1--13, 2017.


\subsubsection*{2016}

\item Evangelia Kasapaki, {\bf Martin Schoeberl }, Rasmus Bo S{\o}rensen, Christian T. M\"uller, Kees Goossens, and Jens Spars{\o}.
 Argo: A Real-Time Network-on-Chip Architecture with an Efficient GALS Implementation.
 \emph{IEEE Transactions on Very Large Scale Integration (VLSI) Systems} 24:479--492, 2016.


\subsubsection*{2015}

\item {\bf Martin Schoeberl }, Sahar Abbaspour, Benny Akesson, Neil Audsley, Raffaele Capasso, Jamie Garside, Kees Goossens, Sven Goossens, Scott Hansen, Reinhold Heckmann, Stefan Hepp, Benedikt Huber, Alexander Jordan, Evangelia Kasapaki, Jens Knoop, Yonghui Li, Daniel Prokesch, Wolfgang Puffitsch, Peter Puschner, Andr\'{e} Rocha, Cl\'{a}udio Silva, Jens Spars{\o}, and Alessandro Tocchi.
 T-CREST: Time-predictable Multi-Core Architecture for Embedded Systems.
 \emph{Journal of Systems Architecture} 61(9):449--471, 2015.


\subsubsection*{2014}


\subsubsection*{2013}

\item {\bf Martin Schoeberl }, Benedikt Huber, and Wolfgang Puffitsch.
 Data cache organization for accurate timing analysis.
 \emph{Real-Time Systems} 49(1):1--28, 2013.

\item Flavius Gruian and {\bf Martin Schoeberl }.
 Hardware Support for CSP on a Java Chip-Multiprocessor.
 \emph{Microprocessors and Microsystems} 37(4--5):472--481, 2013.


\subsubsection*{2012}

\item Trevor Harmon, {\bf Martin Schoeberl }, Raimund Kirner, Raymond Klefstad, K.H. (Kane) Kim, and Michael R. Lowry.
 Fast, Interactive Worst-Case Execution Time Analysis with Back-Annotation.
 \emph{IEEE Transactions on Industrial Informatics} 8:366--377, 2012.

\item Fadi Meawad, Karthik Iyer, {\bf Martin Schoeberl }, and Jan Vitek.
 Micro-transactions for concurrent data structures.
 \emph{Concurrency and Computation: Practice and Experience} 2012.

\item Benedikt Huber, Wolfgang Puffitsch, and {\bf Martin Schoeberl }.
 Worst-case execution time analysis driven object cache design.
 \emph{Concurrency and Computation: Practice and Experience} 24(8):753--771, 2012.

\item Anders P. Ravn and {\bf Martin Schoeberl }.
 Safety-Critical Java with Cyclic Executives on Chip-Multiprocessors.
 \emph{Concurrency and Computation: Practice and Experience} 24:772--788, 2012.


\subsubsection*{2011}

\item {\bf Martin Schoeberl }, Stephan Korsholm, Tomas Kalibera, and Anders P. Ravn.
 A Hardware Abstraction Layer in Java.
 \emph{ACM Trans. Embed. Comput. Syst.} 10(4):42:1--42:40, 2011.


\subsubsection*{2010}

\item Christof Pitter and {\bf Martin Schoeberl }.
 A Real-Time Java Chip-Multiprocessor.
 \emph{ACM Trans. Embed. Comput. Syst.} 10(1):9:1--34, 2010.

\item {\bf Martin Schoeberl } and Wolfgang Puffitsch.
 Nonblocking real-time garbage collection.
 \emph{ACM Trans. Embed. Comput. Syst.} 10(1):6:1--28, 2010.

\item {\bf Martin Schoeberl }.
 Scheduling of Hard Real-Time Garbage Collection.
 \emph{Real-Time Systems} 45(3):176--213, 2010.

\item {\bf Martin Schoeberl }, Wolfgang Puffitsch, Rasmus Ulslev Pedersen, and Benedikt Huber.
 Worst-case execution time analysis for a Java processor.
 \emph{Software: Practice and Experience} 40/6:507--542, 2010.


\subsubsection*{2009}

\item {\bf Martin Schoeberl }.
 Time-predictable Computer Architecture.
 \emph{EURASIP Journal on Embedded Systems} vol. 2009, Article ID 758480:17 pages, 2009.

\item Walter Binder, {\bf Martin Schoeberl }, Philippe Moret, and Alex Villazon.
 Cross-profiling for Java processors.
 \emph{Software: Practice and Experience} 39/18:1439--1465, 2009.


\subsubsection*{2008}

\item {\bf Martin Schoeberl }.
 A Java Processor Architecture for Embedded Real-Time Systems.
 \emph{Journal of Systems Architecture} 54/1--2:265--286, 2008.


\end{enumerate}

\subsection*{Reviewed Conference and Workshop Papers}

\begin{enumerate}

\subsubsection*{2020}

\item Mathieu Jan, Mihail Asavoae, {\bf Martin Schoeberl }, and Edward A. Lee.
 Formal Semantics of Predictable Pipelines: a Comparative Study.
 \emph{TODO: ASP-DAC} January, 2020.

\item Emad Jacob Maroun, {\bf Martin Schoeberl }, and Peter Puschner.
 Towards Dual-Issue Single-Path Code.
 \emph{2020 IEEE 23rd International Symposium on Real-Time Distributed Computing (ISORC)} 176--183, 2020.


\subsubsection*{2019}

\item Eleftherios Kyriakakis, Jens Spars{\o}, and {\bf Martin Schoeberl }.
 InterNoC: Unified Deterministic Communication For Distributed NoC-based Many-Core.
 \emph{13th Junior Researcher Workshop on Real-Time Computing} November, 2019.

\item Mihail Asavoae, Imane Haur, Mathieu Jan, Belgacem Ben Hedia, and {\bf Martin Schoeberl }.
 Towards Formal co-Validation of Hardware and Software Timing Models of CPS.
 \emph{Model-Based Design of Cyber Physical Systems (CyPhy'19)} October, 2019.

\item {\bf Martin Schoeberl }.
 Multicore Models of Communication for Cyber-Physical Systems.
 \emph{Model-Based Design of Cyber Physical Systems (CyPhy'19)} October, 2019.

\item {\bf Martin Schoeberl }, Luca Pezzarossa, and Jens Spars{\o}.
 S4NOC: a Minimalistic Network-on-Chip for Real-Time Multicores.
 \emph{Proceedings of the 12th International Workshop on Network on Chip Architectures} 8:1--8:6, October, 2019.

\item Marten Lohstroh, {\bf Martin Schoeberl }, Mathieu Jan, Edward Wang, and Edward A. Lee.
 Programs with Ironclad Timing Guarantees: Work-in-progress.
 \emph{Proceedings of the International Conference on Embedded Software Companion} 1:1--1:2, New York, New York, October, 2019.

\item Marten Lohstroh, {\bf Martin Schoeberl }, Andr{\'e}s Goens, Armin Wasicek, Christopher Gill, Marjan Sirjani, and Edward A. Lee.
 Actors Revisited for Time-Critical Systems.
 \emph{Proceedings of the 56th Annual Design Automation Conference 2019} 152:1--152:4, Las Vegas, NV, USA, June, 2019.

\item {\bf Martin Schoeberl }, Luca Pezzarossa, and Jens Spars{\o}.
 A Minimal Network Interface for a Simple Network-on-Chip.
 \emph{Architecture of Computing Systems - ARCS 2019} 295--307, May, 2019.

\item Oktay Baris, Shibarchi Majumder, T{\'o}rur Biskopst{\o} Str{\o}m, Anders la Cour-Harbo, Jens Spars{\o}, Thomas Bak, and {\bf Martin Schoeberl }.
 Demonstration of a Time-predictable Flight Controller on a Multicore Processor.
 \emph{Proceedings of the 22nd IEEE International Symposium on Real-time Computing (ISORC)} 95--96, May, 2019.

\item {\bf Martin Schoeberl } and {Morten Borup} Petersen.
 Leros: The return of the accumulator machine.
 \emph{Architecture of Computing Systems - ARCS 2019 - 32nd International Conference, Proceedings} 115--127, May, 2019.

\item Maja Lund, Luca Pezzarossa, Jens Spars{\o}, and {\bf Martin Schoeberl }.
 A Time-predictable TTEthernet Node.
 \emph{2019 IEEE 22nd International Symposium on Real-Time Computing (ISORC)} 229--233, May, 2019.

\item {\bf Martin Schoeberl }, Benjamin Rouxel, and Isabelle Puaut.
 A Time-predictable Branch Predictor.
 \emph{Proceedings of the 34th ACM/SIGAPP Symposium on Applied Computing} 607--616, Limassol, Cyprus, April, 2019.

\item {\bf Martin Schoeberl }, T{\'o}rur Biskopst{\o} Str{\o}m, Oktay Baris, and Jens Spars\o{}.
 Scratchpad Memories with Ownership.
 \emph{2019 Design, Automation and Test in Europe Conference Exhibition (DATE)} 1216--1221, March, 2019.

\item Christos Gkiokas and {\bf Martin Schoeberl }.
 A Fault-Tolerant Time-Predictable Processor.
 \emph{2019 IEEE Nordic Circuits and Systems Conference (NORCAS): NORCHIP and International Symposium of System-on-Chip (SoC)} 1--6, 2019.

\item Eleftherios Kyriakakis, Jens Spars{\o}, and {\bf Martin Schoeberl }.
 Implementing time-triggered communication over a standard ethernet switch.
 \emph{Proceedings of the Fog-IoT Workshop 2019} 21--25, 2019.


\subsubsection*{2018}

\item Morten B. Petersen, Anthon V. Riber, Simon T. Andersen, and {\bf Martin Schoeberl }.
 Time-Predictable Distributed Shared Memory for Multi-Core Processors.
 \emph{2018 IEEE Nordic Circuits and Systems Conference (NORCAS): NORCHIP and International Symposium of System-on-Chip (SoC)} 1-7, October, 2018.

\item Hammond Pearce, Partha Roop, Morteza Biglari-Abhari, and {\bf Martin Schoeberl }.
 Faster Function Blocks for Precision Timed Industrial Automation.
 \emph{2018 IEEE 21st International Symposium on Real-Time Distributed Computing (ISORC)} 67-74, May, 2018.

\item {\bf Martin Schoeberl } and Rasmus Ulslev Pedersen.
 tpIP: A Time-Predictable TCP/IP Stack for Cyber-Physical Systems.
 \emph{2018 IEEE 21st International Symposium on Real-Time Distributed Computing (ISORC)} 75--82, May, 2018.

\item T{\'o}rur Biskopst{\o} Str{\o}m and {\bf Martin Schoeberl }.
 Hardlock: A Concurrent Real-Time Multicore Locking Unit.
 \emph{2018 IEEE 21st International Symposium on Real-Time Distributed Computing (ISORC)} 9--16, May, 2018.

\item {\bf Martin Schoeberl }.
 Design of a Time-predictable Multicore Processor: The T-CREST Project.
 \emph{2018 Design, Automation Test in Europe Conference Exhibition (DATE)} 909--912, March, 2018.

\item {\bf Martin Schoeberl }.
 One-Way Shared Memory.
 \emph{2018 Design, Automation and Test in Europe Conference Exhibition (DATE)} 269--272, March, 2018.

\item Eleftherios Kyriakakis, Jens Spars{\o}, and {\bf Martin Schoeberl }.
 Hardware Assisted Clock Synchronization with the IEEE 1588-2008 Precision Time Protocol.
 \emph{Proceedings of the 26th International Conference on Real-Time Networks and Systems} 51--60, 2018.

\item {\bf Martin Schoeberl }.
 Lipsi: Probably the Smallest Processor in the World.
 \emph{Architecture of Computing Systems -- ARCS 2018} 18--30, 2018.


\subsubsection*{2017}

\item Henrik Enggaard Hansen, Emad Jacob Maroun, Andreas Toftegaard Kristensen, Jimmi Marquart, and {\bf Martin Schoeberl }.
 A Shared Scratchpad Memory with Synchronization Support.
 \emph{2017 IEEE Nordic Circuits and Systems Conference (NORCAS): NORCHIP and International Symposium of System-on-Chip (SoC)} 1-6, October, 2017.

\item {\bf Martin Schoeberl } and Jens Spars{\o}.
 Timing Organization of a Real-Time Multicore Processor.
 \emph{2017 New Generation of CAS (NGCAS)} 89--92, September, 2017.

\item Bekim Cilku, Wolfgang Puffitsch, Daniel Prokesch, {\bf Martin Schoeberl }, and Peter Puschner.
 Improving Performance of Single-path Code Through a Time-predictable Memory Hierarchy.
 \emph{Proceedings of the 20th IEEE International Symposium on Real-Time Distributed Computing (ISORC 2017)} 76--83, May, 2017.

\item Luca Pezzarossa, {\bf Martin Schoeberl }, and Jens Spars{\o}.
 A Controller for Dynamic Partial Reconfiguration in FPGA-Based Real-Time Systems.
 \emph{2017 IEEE 20th International Symposium on Real-Time Distributed Computing (ISORC)} 92--100, May, 2017.

\item Daniel Sanz Ausin, Luca Pezzarossa, and {\bf Martin Schoeberl }.
 Real-Time Audio Processing on the T-CREST Multicore Platform.
 \emph{2017 IEEE 11th International Symposium on Embedded Multicore/Many-core Systems-on-Chip (MCSoC)} 2017.

\item Luca Pezzarossa, Andreas Toftegaard Kristensen, {\bf Martin Schoeberl }, and Jens Spars{\o}.
 Can Real-Time Systems Benefit from Dynamic Partial Reconfiguration?.
 \emph{2017 IEEE Nordic Circuits and Systems Conference (NORCAS): NORCHIP and International Symposium of System-on-Chip (SoC)} 2017.

\item {\bf Martin Schoeberl }, Bekim Cilku, Daniel Prokesch, and Peter Puschner.
 Best Practice for Caching of Single-Path Code.
 \emph{17th International Workshop on Worst-Case Execution Time Analysis (WCET 2017)} 1--12, 2017.


\subsubsection*{2016}

\item Alexander Jordan, Sahar Abbaspour, and {\bf Martin Schoeberl }.
 A Software Managed Stack Cache for Real-Time Systems.
 \emph{Proceedings of the 24th International Conference on Real-Time Networks and Systems (RTNS 2016)} 319--326, October, 2016.

\item Rasmus Bo S{\o}rensen, {\bf Martin Schoeberl }, and Jens Spars{\o}.
 State-based Communication on Time-predictable Multicore Processors.
 \emph{Proceedings of the 24th International Conference on Real-Time Networks and Systems (RTNS 2016)} 225--234, October, 2016.

\item Florian Kluge, {\bf Martin Schoeberl }, and Theo Ungerer.
 Support for the Logical Execution Time Model on a Time-predictable Multicore Processor.
 \emph{14th International Workshop on Real-Time Networks} July, 2016.

\item Luca Pezzarossa, {\bf Martin Schoeberl }, and Jens Spars{\o}.
 Reconfiguration in FPGA-based multi-core platforms for hard real-time applications.
 \emph{11th International Symposium on Reconfigurable Communication-centric Systems-on-Chip (ReCoSoC 2016)} 1--8, June, 2016.

\item {\bf Martin Schoeberl } and Carsten Nielsen.
 A Stack Cache for Real-Time Systems.
 \emph{Proceedings of the 19th IEEE Symposium on Real-time Distributed Computing (ISORC 2016)} 150--157, May, 2016.

\item Wolfgang Puffitsch and {\bf Martin Schoeberl }.
 Time-Predictable Virtual Memory.
 \emph{Proceedings of the 19th IEEE Symposium on Real-time Distributed Computing (ISORC 2016)} 158--165, May, 2016.

\item {\bf Martin Schoeberl }.
 Lessons learned from the EU project T-CREST.
 \emph{Design, Automation Test in Europe Conference Exhibition (DATE 2016)} 870--875, March, 2016.

\item Heiko Falk, Sebastian Altmeyer, Peter Hellinckx, Bj{\"o}rn Lisper, Wolfgang Puffitsch, Christine Rochange, {\bf Martin Schoeberl }, Rasmus Bo S{\o}rensen, Peter W{\"a}gemann, and Simon Wegener.
 TACLeBench: A Benchmark Collection to Support Worst-Case Execution Time Research.
 \emph{16th International Workshop on Worst-Case Execution Time Analysis (WCET 2016)} 2:1--2:10, 2016.


\subsubsection*{2015}

\item Wolfgang Puffitsch, Rasmus Bo S\o{}rensen, and {\bf Martin Schoeberl }.
 Time-Division Multiplexing vs Network Calculus: A Comparison.
 \emph{Proceedings of the 23th International Conference on Real-Time and Network Systems (RTNS 2015)} November, 2015.

\item Luca Pezzarossa, Rasmus Bo S{\o}rensen, {\bf Martin Schoeberl }, and Jens Spars{\o}.
 Interfacing Hardware Accelerators to a Time-Division Multiplexing Network-on-Chip.
 \emph{Proc. of the 1st Nordic Circuits and Systems Conference (NORCAS 2015)} October, 2015.

\item Stephan E. Korsholm, {\bf Martin Schoeberl }, and Wolfgang Puffitsch.
 Safety-Critical Java on a Time-Predictable Processor.
 \emph{Proceedings of the 13th International Workshop on Java Technologies for Real-Time and Embedded Systems (JTRES 2015)} October, 2015.

\item T{\'o}rur Biskopst{\o} Str{\o}m and {\bf Martin Schoeberl }.
 Multiprocessor Priority Ceiling Emulation for Safety-Critical Java.
 \emph{Proceedings of the 13th International Workshop on Java Technologies for Real-Time and Embedded Systems (JTRES 2015)} October, 2015.

\item {\bf Martin Schoeberl }.
 Scala for Real-Time Systems?.
 \emph{Proceedings of the 13th International Workshop on Java Technologies for Real-Time and Embedded Systems (JTRES 2015)} October, 2015.

\item Carsten Nielsen and {\bf Martin Schoeberl }.
 Stack Caching Using Split Data Caches.
 \emph{Proceedings of the 11th Workshop on Software Technologies for Embedded and Ubiquitous Systems (SEUS 2015)} 36--43, April, 2015.

\item T{\'o}rur Biskopst{\o} Str{\o}m and {\bf Martin Schoeberl }.
 Hardware Locks with Priority Ceiling Emulation for a Java Chip-Multiprocessor.
 \emph{Proceedings of the 17th IEEE Symposium on Real-time Distributed Computing (ISORC 2015)} 268--271, April, 2015.

\item Marco Ziccardi, {\bf Martin Schoeberl }, and Tullio Vardanega.
 A Time-Composable Operating System for the Patmos Processor.
 \emph{The 30th ACM/SIGAPP Symposium On Applied Computing, Embedded Systems Track} April, 2015.

\item Rasmus Bo S{\o}rensen, Wolfgang Puffitsch, {\bf Martin Schoeberl }, and Jens Spars{\o}.
 Message Passing on a Time-predictable Multicore Processor.
 \emph{Proceedings of the 17th IEEE Symposium on Real-time Distributed Computing (ISORC 2015)} 51--59, April, 2015.

\item {\bf Martin Schoeberl }, Rasmus Bo S{\o}rensen, and Jens Spars{\o}.
 Models of Communication for Multicore Processors.
 \emph{Proceedings of the 11th Workshop on Software Technologies for Embedded and Ubiquitous Systems (SEUS 2015)} 44--51, April, 2015.

\item Luca Pezzarossa, {\bf Martin Schoeberl }, and Jens Spars{\o}.
 Towards Utilizing Reconfigurable Shared Resources in Multi-Core Hard Real-Time Systems.
 \emph{9th Junior Researcher Workshop on Real-Time Computing JRWRTC 2015} 21--24, 2015.


\subsubsection*{2014}

\item {\bf Martin Schoeberl }, Andreas Engelbredt Dalsgaard, Ren\'{e} Rydhof Hansen, Stephan E. Korsholm, Anders P. Ravn, Juan Ricardo Rios Rivas, T\'{o}rur Biskopst{\o} Str{\o}m, and Hans S{\o}ndergaard.
 Certifiable Java for Embedded Systems.
 \emph{Proceedings of the 12th International Workshop on Java Technologies for Real-Time and Embedded Systems (JTRES 2014)} 10--19, October, 2014.

\item Benedikt Huber, Stefan Hepp, and {\bf Martin Schoeberl }.
 Scope-based Method Cache Analysis.
 \emph{Proceedings of the 14th International Workshop on Worst-Case Execution Time Analysis (WCET 2014)} 73--82, July, 2014.

\item {\bf Martin Schoeberl }, David VH Chong, Wolfgang Puffitsch, and Jens Spars{\o}.
 A Time-predictable Memory Network-on-Chip.
 \emph{Proceedings of the 14th International Workshop on Worst-Case Execution Time Analysis (WCET 2014)} 53--62, July, 2014.

\item Jack Whitham and {\bf Martin Schoeberl }.
 WCET-Based Comparison of an Instruction Scratchpad and a Method Cache.
 \emph{Proceedings of the 10th Workshop on Software Technologies for Embedded and Ubiquitous Systems (SEUS 2014)} June, 2014.

\item Juan Ricardo Rios and {\bf Martin Schoeberl }.
 An Evaluation of Safety-Critical Java on a Java Processor.
 \emph{Proceedings of the 10th Workshop on Software Technologies for Embedded and Ubiquitous Systems (SEUS 2014)} June, 2014.

\item Philipp Degasperi, Stefan Hepp, Wolfgang Puffitsch, and {\bf Martin Schoeberl }.
 A Method Cache for Patmos.
 \emph{Proceedings of the 17th IEEE Symposium on Object/Component/Service-oriented Real-time Distributed Computing (ISORC 2014)} 100--108, June, 2014.

\item {\bf Martin Schoeberl }, Cl\'{a}udio Silva, and Andr\'{e} Rocha.
 T-CREST: A Time-predictable Multi-Core Platform for Aerospace Applications.
 \emph{Proceedings of Data Systems In Aerospace (DASIA 2014)} June, 2014.

\item Juan Ricardo Rios and {\bf Martin Schoeberl }.
 Reusable Libraries for Safety-Critical Java.
 \emph{Proceedings of the 17th IEEE Symposium on Object/Component/Service-oriented Real-time Distributed Computing (ISORC 2014)} 188--197, June, 2014.


\subsubsection*{2013}

\item T{\'o}rur Biskopst{\o} Str{\o}m, Wolfgang Puffitsch, and {\bf Martin Schoeberl }.
 Chip-Multiprocessor Hardware Locks for Safety-Critical Java.
 \emph{Proceedings of the 11th International Workshop on Java Technologies for Real-Time and Embedded Systems (JTRES 2013)} 38--46, October, 2013.

\item Alexander Jordan, Florian Brandner, and {\bf Martin Schoeberl }.
 Static Analysis of Worst-case Stack Cache Behavior.
 \emph{Proceedings of the 21st International Conference on Real-Time Networks and Systems (RTNS 2013)} 55--64, Sophia Antipolis, France, 2013.

\item Edgar Lakis and {\bf Martin Schoeberl }.
 An SDRAM Controller for Real-Time Systems.
 \emph{Proceedings of the 9th Workshop on Software Technologies for Embedded and Ubiquitous Systems} 2013.

\item Jens Spars{\o}, Evangelia Kasapaki, and {\bf Martin Schoeberl }.
 An Area-efficient Network Interface for a TDM-based Network-on-Chip.
 \emph{Proceedings of the Conference on Design, Automation and Test in Europe} 1044--1047, Grenoble, France, 2013.

\item Sahar Abbaspour, Florian Brandner, and {\bf Martin Schoeberl }.
 A Time-predictable Stack Cache.
 \emph{Proceedings of the 9th Workshop on Software Technologies for Embedded and Ubiquitous Systems} 2013.


\subsubsection*{2012}

\item Rasmus Bo S{\o}rensen, {\bf Martin Schoeberl }, and Jens Spars{\o}.
 A Light-Weight Statically Scheduled Network-on-Chip.
 \emph{Proceedings of the 29th Norchip Conference} November, 2012.

\item Florian Brandner and {\bf Martin Schoeberl }.
 Static Routing in Symmetric Real-Time Network-on-Chips.
 \emph{Proceedings of the 20th International Conference on Real-Time and Network Systems (RTNS 2012)} 61--70, November, 2012.

\item Wolfgang Puffitsch and {\bf Martin Schoeberl }.
 On the Scalability of Time-predictable Chip-Multiprocessing.
 \emph{Proceedings of the 10th International Workshop on Java Technologies for Real-Time and Embedded Systems (JTRES 2012)} 98--104, October, 2012.

\item Andreas E. Dalsgaard, Ren\'{e} Rydhof Hansen, and {\bf Martin Schoeberl }.
 Private Memory Allocation Analysis for Safety-Critical Java.
 \emph{Proceedings of the 10th International Workshop on Java Technologies for Real-Time and Embedded Systems (JTRES 2012)} 9--17, October, 2012.

\item Juan Ricardo Rios, Kelvin Nilsen, and {\bf Martin Schoeberl }.
 Patterns for Safety-Critical Java Memory Usage.
 \emph{Proceedings of the 10th International Workshop on Java Technologies for Real-Time and Embedded Systems (JTRES 2012)} 1--8, October, 2012.

\item T{\'o}rur Biskopst{\o} Str{\o}m and {\bf Martin Schoeberl }.
 A Desktop 3D Printer in Safety-Critical Java.
 \emph{Proceedings of the 10th International Workshop on Java Technologies for Real-Time and Embedded Systems (JTRES 2012)} 72--79, October, 2012.

\item {\bf Martin Schoeberl } and Juan Ricardo Rios.
 Safety-Critical Java on a Java Processor.
 \emph{Proceedings of the 10th International Workshop on Java Technologies for Real-Time and Embedded Systems (JTRES 2012)} 54--61, October, 2012.

\item {\bf Martin Schoeberl }.
 Is Time Predictability Quantifiable?.
 \emph{International Conference on Embedded Computer Systems (SAMOS 2012)} July, 2012.

\item {\bf Martin Schoeberl }, Florian Brandner, Jens Spars{\o}, and Evangelia Kasapaki.
 A Statically Scheduled Time-Division-Multiplexed Network-on-Chip for Real-Time Systems.
 \emph{Proceedings of the 6th International Symposium on Networks-on-Chip (NOCS)} 152--160, May, 2012.

\item Juan Ricardo Rios and {\bf Martin Schoeberl }.
 Hardware Support for Safety-Critical Java Scope Checks.
 \emph{Proceedings of the 15th IEEE International Symposium on Object/component/service-oriented Real-time distributed Computing (ISORC 2012)} 31--38, April, 2012.

\item Stefan Hepp and {\bf Martin Schoeberl }.
 Worst-Case Execution Time Based Optimization of Real-Time Java Programs.
 \emph{Proceedings of the 15th IEEE International Symposium on Object/component/service-oriented Real-time distributed Computing (ISORC 2012)} 64--70, April, 2012.


\subsubsection*{2011}

\item {\bf Martin Schoeberl }.
 Leros: A Tiny Microcontroller for FPGAs.
 \emph{Proceedings of the 21st International Conference on Field Programmable Logic and Applications (FPL 2011)} 10--14, September, 2011.

\item James Caska and {\bf Martin Schoeberl }.
 Java Dust: How Small Can Embedded Java Be?.
 \emph{Proceedings of the 9th International Workshop on Java Technologies for Real-Time and Embedded Systems (JTRES 2011)} 125--129, September, 2011.

\item Fadi Meawad, Karthik Iyer, {\bf Martin Schoeberl }, and Jan Vitek.
 Real-Time Wait-free Queues using Micro-Transactions.
 \emph{Proceedings of the 9th International Workshop on Java Technologies for Real-Time and Embedded Systems (JTRES 2011)} 1--10, September, 2011.

\item Andy Wellings and {\bf Martin Schoeberl }.
 User-Defined Clocks in the Real-Time Specification for Java.
 \emph{Proceedings of the 9th International Workshop on Java Technologies for Real-Time and Embedded Systems (JTRES 2011)} 74--81, September, 2011.

\item {\bf Martin Schoeberl }.
 Memory Management for Safety-Critical Java.
 \emph{Proceedings of the 9th International Workshop on Java Technologies for Real-Time and Embedded Systems (JTRES 2011)} 47--53, September, 2011.

\item {\bf Martin Schoeberl }.
 ejIP: A TCP/IP Stack for Embedded Java.
 \emph{Proceedings of the 9th International Conference on the Principles and Practice of Programming in Java (PPPJ 2011)} August, 2011.

\item Aibek Sarimbekov, Andreas Sewe, Walter Binder, Philippe Moret, {\bf Martin Schoeberl }, and Mira Mezini.
 Portable and Accurate Collection of Calling-Context-Sensitive Bytecode Metrics for the Java Virtual Machine.
 \emph{Proceedings of the 9th International Conference on the Principles and Practice of Programming in Java (PPPJ 2011)} August, 2011.

\item Reinhard von Hanxleden, Niklas Holsti, Bj{\"o}rn Lisper, Erhard Ploedereder, Reinhard Wilhelm, Armelle Bonenfant, Hugues Casse, Sven B{\"u}nte, Wolfgang Fellger, Sebastian Gepperth, Jan Gustafsson, Benedikt Huber, Nazrul Mohammad Islam, Daniel K{\"a}stner, Raimund Kirner, Laura Kovacs, Felix Krause, Marianne de Michiel, Mads Christian Olesen, Adrian Prantl, Wolfgang Puffitsch, Christine Rochange, {\bf Martin Schoeberl }, Simon Wegener, Michael Zolda, and Jakob Zwirchmayr.
 WCET Tool Challenge 2011: Report.
 \emph{Proceedings of the 11th International Workshop on Worst-Case Execution Time (WCET) Analysis} July, 2011.

\item Christian Stoif, {\bf Martin Schoeberl }, Benito Liccardi, and Jan Haase.
 Hardware Synchronization for Embedded Multi-Core Processors.
 \emph{Proceedings of the 2011 IEEE International Symposium on Circuits and Systems (ISCAS 2011)} May, 2011.

\item {\bf Martin Schoeberl }, Walter Binder, and Alex Villazon.
 Design Space Exploration of Object Caches with Cross-Profiling.
 \emph{Proceedings of the 14th IEEE International Symposium on Object/component/service-oriented Real-time distributed Computing (ISORC 2011)} 213--221, March, 2011.

\item {\bf Martin Schoeberl }, Pascal Schleuniger, Wolfgang Puffitsch, Florian Brandner, Christian W. Probst, Sven Karlsson, and Tommy Thorn.
 Towards a Time-predictable Dual-Issue Microprocessor: The Patmos Approach.
 \emph{First Workshop on Bringing Theory to Practice: Predictability and Performance in Embedded Systems (PPES 2011)} 11--20, March, 2011.

\item {\bf Martin Schoeberl }.
 A Time-predictable Object Cache.
 \emph{Proceedings of the 14th IEEE International Symposium on Object/component/service-oriented Real-time distributed Computing (ISORC 2011)} 99--105, March, 2011.


\subsubsection*{2010}

\item Flavius Gruian and {\bf Martin Schoeberl }.
 NoC-based CSP Support for a Java Chip Multiprocessor.
 \emph{Proceedings of the 28th Norchip Conference} November, 2010.

\item {\bf Martin Schoeberl }.
 Time-predictable Chip-Multiprocessor Design.
 \emph{Asilomar Conference on Signals, Systems, and Computers} November, 2010.

\item {\bf Martin Schoeberl }, Christopher Brooks, and Edward A. Lee.
 Code Generation for Embedded Java with Ptolemy.
 \emph{Proceedings of the 8th IFIP Workshop on Software Technologies for Future Embedded and Ubiquitous Systems (SEUS 2010)} October, 2010.

\item {\bf Martin Schoeberl }, Thomas B. Preusser, and Sascha Uhrig.
 The Embedded Java Benchmark Suite JemBench.
 \emph{Proceedings of the 8th International Workshop on Java Technologies for Real-Time and Embedded Systems (JTRES 2010)} 120--127, Prague, Czech Republic, August, 2010.

\item {\bf Martin Schoeberl } and Peter Hilber.
 Design and Implementation of Real-Time Transactional Memory.
 \emph{Proceedings of the 20th International Conference on Field Programmable Logic and Applications (FPL 2010)} 279--284, August, 2010.

\item {\bf Martin Schoeberl }, Florian Brandner, and Jan Vitek.
 RTTM: Real-Time Transactional Memory.
 \emph{Proceedings of the 25th ACM Symposium on Applied Computing (SAC 2010)} 326--333, March, 2010.

\item Rasmus Ulslev Pedersen and {\bf Martin Schoeberl }.
 Object oriented machine learning with a multicore real-time Java processor: short paper.
 \emph{Proceedings of the 8th International Workshop on Java Technologies for Real-time and Embedded Systems (JTRES 2010)} 76--78, Prague, Czech Republic, 2010.

\item Tomas Kalibera, Pavel Parizek, Michal Malohlava, and {\bf Martin Schoeberl }.
 Exhaustive testing of safety critical Java.
 \emph{Proceedings of the 8th International Workshop on Java Technologies for Real-time and Embedded Systems (JTRES 2010)} 164--174, Prague, Czech Republic, 2010.

\item Benedikt Huber, Wolfgang Puffitsch, and {\bf Martin Schoeberl }.
 WCET Driven Design Space Exploration of an Object Cache.
 \emph{Proceedings of the 8th International Workshop on Java Technologies for Real-time and Embedded Systems (JTRES 2010)} 26--35, Prague, Czech Republic, 2010.

\item Anders P. Ravn and {\bf Martin Schoeberl }.
 Cyclic executive for safety-critical Java on chip-multiprocessors.
 \emph{Proceedings of the 8th International Workshop on Java Technologies for Real-time and Embedded Systems (JTRES 2010)} 63--69, Prague, Czech Republic, 2010.

\item Wolfgang Puffitsch, Benedikt Huber, and {\bf Martin Schoeberl }.
 Worst-Case Analysis of Heap Allocations.
 \emph{Proceedings of the 4th International Symposium On Leveraging Applications of Formal Methods, Verification and Validation (ISoLA 2010)} 464--478, 2010.


\subsubsection*{2009}

\item {\bf Martin Schoeberl }, Wolfgang Puffitsch, and Benedikt Huber.
 Towards Time-predictable Data Caches for Chip-Multiprocessors.
 \emph{Proceedings of the Seventh IFIP Workshop on Software Technologies for Future Embedded and Ubiquitous Systems (SEUS 2009)} 180--191, November, 2009.

\item {\bf Martin Schoeberl }, Peter Puschner, and Raimund Kirner.
 A Single-Path Chip-Multiprocessor System.
 \emph{Proceedings of the Seventh IFIP Workshop on Software Technologies for Future Embedded and Ubiquitous Systems (SEUS 2009)} 47--57, November, 2009.

\item Rasmus Ulslev Pedersen and {\bf Martin Schoeberl }.
 Educational Case Studies with an Open Source Embedded Real-Time Java Processor.
 \emph{Proceedings of the 2009 Workshop on Embedded Systems Education (WESE 2009)} 71--77, Grenoble, France, October, 2009.

\item Stephen A. Edwards, Sungjun Kim, Edward A. Lee, Isaac Liu, Hiren D. Patel, and {\bf Martin Schoeberl }.
 A Disruptive Computer Design Idea: Architectures with Repeatable Timing.
 \emph{Proceedings of IEEE International Conference on Computer Design (ICCD 2009)} October, 2009.

\item {\bf Martin Schoeberl }, Peter Puschner, and Raimund Kirner.
 Single-Path Programming on a Chip-Multiprocessor System.
 \emph{Workshop on Reconciling Performance with Predictability (RePP)} October, 2009.

\item Jack Whitham, Neil Audsley, and {\bf Martin Schoeberl }.
 Using Hardware Methods to Improve Time-predictable Performance in Real-time Java Systems.
 \emph{Proceedings of the 7th International Workshop on Java Technologies for Real-time and Embedded Systems (JTRES 2009)} 130--139, September, 2009.

\item Philippe Moret, Walter Binder, Alex Villazon, Danilo Ansaloni, and {\bf Martin Schoeberl }.
 Locating Performance Bottlenecks in Embedded Java Software with Calling-Context Cross-Profiling.
 \emph{Proceedings of the 6th International Conference on the Quantitative Evaluation of SysTems (QEST 2009)} 107--108, September, 2009.

\item {\bf Martin Schoeberl }, Walter Binder, Philippe Moret, and Alex Villazon.
 Design Space Exploration for Java Processors with Cross-Profiling.
 \emph{Proceedings of the 6th International Conference on the Quantitative Evaluation of SysTems (QEST 2009)} 109--118, September, 2009.

\item Philippe Moret, Walter Binder, {\bf Martin Schoeberl }, Alex Villazon, and Danilo Ansaloni.
 Analyzing Performance and Dynamic Behavior of Embedded Java Software with Calling-Context Cross-Profiling.
 \emph{Proceedings of the 7th International Conference on the Principles and Practice of Programming in Java (PPPJ 2009)} 121--124, August, 2009.

\item {\bf Martin Schoeberl } and Peter Puschner.
 Is Chip-Multiprocessing the End of Real-Time Scheduling?.
 \emph{Proceedings of the 9th International Workshop on Worst-Case Execution Time (WCET) Analysis} July, 2009.

\item Benedikt Huber and {\bf Martin Schoeberl }.
 Comparison of Implicit Path Enumeration and Model Checking based WCET Analysis.
 \emph{Proceedings of the 9th International Workshop on Worst-Case Execution Time (WCET) Analysis} 23--34, July, 2009.

\item {\bf Martin Schoeberl }.
 Time-predictable Cache Organization.
 \emph{Proceedings of the First International Workshop on Software Technologies for Future Dependable Distributed Systems (STFSSD 2009)} 11--16, March, 2009.

\item Andy Wellings and {\bf Martin Schoeberl }.
 Thread-local Scope Caching for Real-time Java.
 \emph{Proceedings of the 12th IEEE International Symposium on Object/component/service-oriented Real-time distributed Computing (ISORC 2009)} 275--282, March, 2009.

\item Thomas Henties, James J. Hunt, Doug Locke, Kelvin Nilsen, {\bf Martin Schoeberl }, and Jan Vitek.
 Java for Safety-Critical Applications.
 \emph{2nd International Workshop on the Certification of Safety-Critical Software Controlled Systems (SafeCert 2009)} March, 2009.

\item Florian Brandner, Tommy Thorn, and {\bf Martin Schoeberl }.
 Embedded JIT Compilation with CACAO on YARI.
 \emph{Proceedings of the 12th IEEE International Symposium on Object/component/service-oriented Real-time distributed Computing (ISORC 2009)} 63--70, March, 2009.

\item Stephen A. Edwards, Sungjun Kim, Edward A. Lee, Hiren D. Patel, and {\bf Martin Schoeberl }.
 Reconciling Repeatable Timing with Pipelining and Memory Hierarchy.
 \emph{Workshop on Reconciling Performance with Predictability (RePP)} 2009.


\subsubsection*{2008}

\item Walter Binder, Alex Villazon, {\bf Martin Schoeberl }, and Philippe Moret.
 Cache-aware Cross-profiling for Java Processors.
 \emph{Proceedings of the 2008 international conference on Compilers, architecture, and synthesis for embedded systems (CASES 2008)} 127--136, October, 2008.

\item Walter Binder, {\bf Martin Schoeberl }, Philippe Moret, and Alex Villazon.
 Cross-Profiling for Embedded Java Processors.
 \emph{Proceedings of the 5th International Conference on the Quantitative Evaluation of SysTems (QEST 2008)} 287--296, September, 2008.

\item Wolfgang Puffitsch and {\bf Martin Schoeberl }.
 Non-Blocking Root Scanning for Real-Time Garbage Collection.
 \emph{Proceedings of the 6th International Workshop on Java Technologies for Real-time and Embedded Systems (JTRES 2008)} 68--76, September, 2008.

\item {\bf Martin Schoeberl } and Wolfgang Puffitsch.
 Non-blocking Object Copy for Real-Time Garbage Collection.
 \emph{Proceedings of the 6th International Workshop on Java Technologies for Real-time and Embedded Systems (JTRES 2008)} 77--84, September, 2008.

\item {\bf Martin Schoeberl }.
 Application Experiences with a Real-Time Java Processor.
 \emph{Proceedings of the 17th IFAC World Congress} 9320--9325, July, 2008.

\item Peter Puschner and {\bf Martin Schoeberl }.
 On Composable System Timing, Task Timing, and WCET Analysis.
 \emph{Proceedings of the 8th International Workshop on Worst-Case Execution Time (WCET) Analysis} 91--101, July, 2008.

\item Christof Pitter and {\bf Martin Schoeberl }.
 Performance Evaluation of a Java Chip-Multiprocessor.
 \emph{Proceedings of the 3rd IEEE Symposium on Industrial Embedded Systems (SIES 2008)} 34--42, June, 2008.

\item Trevor Harmon, {\bf Martin Schoeberl }, Raimund Kirner, and Raymond Klefstad.
 Toward Libraries for Real-time Java.
 \emph{Proceedings of the 11th IEEE International Symposium on Object/component/service-oriented Real-time distributed Computing (ISORC 2008)} 458--462, May, 2008.

\item {\bf Martin Schoeberl }, Stephan Korsholm, Christian Thalinger, and Anders P. Ravn.
 Hardware Objects for Java.
 \emph{Proceedings of the 11th IEEE International Symposium on Object/component/service-oriented Real-time distributed Computing (ISORC 2008)} 445--452, May, 2008.

\item Stephan Korsholm, {\bf Martin Schoeberl }, and Anders P. Ravn.
 Interrupt Handlers in Java.
 \emph{Proceedings of the 11th IEEE International Symposium on Object/component/service-oriented Real-time distributed Computing (ISORC 2008)} 453--457, May, 2008.

\item Trevor Harmon, {\bf Martin Schoeberl }, Raimund Kirner, and Raymond Klefstad.
 A Modular Worst-case Execution Time Analysis Tool for Java Processors.
 \emph{Proceedings of the 14th IEEE Real-Time and Embedded Technology and Applications Symposium (RTAS 2008)} 47--57, April, 2008.


\subsubsection*{2007}

\item {\bf Martin Schoeberl }.
 SimpCon - a Simple and Efficient SoC Interconnect.
 \emph{Proceedings of the 15th Austrian Workshop on Microelectronics, Austrochip 2007} October, 2007.

\item Christof Pitter and {\bf Martin Schoeberl }.
 Towards a Java Multiprocessor.
 \emph{Proceedings of the 5th International Workshop on Java Technologies for Real-time and Embedded Systems (JTRES 2007)} 144--151, September, 2007.

\item {\bf Martin Schoeberl }.
 Architecture for Object Oriented Programming Languages.
 \emph{Proceedings of the 5th International Workshop on Java Technologies for Real-time and Embedded Systems (JTRES 2007)} 57--62, September, 2007.

\item {\bf Martin Schoeberl } and Jan Vitek.
 Garbage Collection for Safety Critical Java.
 \emph{Proceedings of the 5th International Workshop on Java Technologies for Real-time and Embedded Systems (JTRES 2007)} 85--93, September, 2007.

\item Wolfgang Puffitsch and {\bf Martin Schoeberl }.
 picoJava-II in an FPGA.
 \emph{Proceedings of the 5th International Workshop on Java Technologies for Real-time and Embedded Systems (JTRES 2007)} 213--221, September, 2007.

\item {\bf Martin Schoeberl }.
 A Time-Triggered Network-on-Chip.
 \emph{International Conference on Field-Programmable Logic and its Applications (FPL 2007)} 377--382, August, 2007.

\item Christof Pitter and {\bf Martin Schoeberl }.
 Time Predictable CPU and DMA Shared Memory Access.
 \emph{International Conference on Field-Programmable Logic and its Applications (FPL 2007)} 317--322, August, 2007.

\item Raimund Kirner and {\bf Martin Schoeberl }.
 Modeling the Function Cache for Worst-Case Execution Time Analysis.
 \emph{Proceedings of the 44rd Design Automation Conference (DAC 2007)} 471--476, June, 2007.

\item {\bf Martin Schoeberl }, Hans Sondergaard, Bent Thomsen, and Anders P. Ravn.
 A Profile for Safety Critical Java.
 \emph{10th IEEE International Symposium on Object and Component-Oriented Real-Time Distributed Computing (ISORC'07)} 94--101, May, 2007.

\item {\bf Martin Schoeberl }.
 Mission Modes for Safety Critical Java.
 \emph{Software Technologies for Embedded and Ubiquitous Systems, 5th {IFIP} {WG} 10.2 International Workshop (SEUS 2007)} 105--113, May, 2007.


\subsubsection*{2006}

\item Rasmus Pedersen and {\bf Martin Schoeberl }.
 An Embedded Support Vector Machine.
 \emph{Proceedings of the Fourth Workshop on Intelligent Solutions in Embedded Systems (WISES 2006)} 79--89, June, 2006.

\item {\bf Martin Schoeberl }.
 Real-Time Garbage Collection for Java.
 \emph{Proceedings of the 9th IEEE International Symposium on Object and Component-Oriented Real-Time Distributed Computing (ISORC 2006)} 424--432, April, 2006.

\item {\bf Martin Schoeberl }.
 A Time Predictable Java Processor.
 \emph{Proceedings of the Design, Automation and Test in Europe Conference (DATE 2006)} 800--805, March, 2006.

\item Rasmus Pedersen and {\bf Martin Schoeberl }.
 Exact Roots for a Real-Time Garbage Collector.
 \emph{Proceedings of the 4th International Workshop on Java Technologies for Real-time and Embedded Systems (JTRES 2006)} 77--84, Paris, France, 2006.

\item {\bf Martin Schoeberl } and Rasmus Pedersen.
 WCET Analysis for a Java Processor.
 \emph{Proceedings of the 4th International Workshop on Java Technologies for Real-time and Embedded Systems (JTRES 2006)} 202--211, Paris, France, 2006.


\subsubsection*{2005}

\item {\bf Martin Schoeberl }.
 Evaluation of a Java Processor.
 \emph{Tagungsband Austrochip 2005} 127--134, October, 2005.

\item {\bf Martin Schoeberl }.
 Design and Implementation of an Efficient Stack Machine.
 \emph{Proceedings of the 12th IEEE Reconfigurable Architecture Workshop (RAW2005)} April, 2005.

\item Flavius Gruian, Per Andersson, Krzysztof Kuchcinski, and {\bf Martin Schoeberl }.
 Automatic Generation of Application-Specific Systems Based on a Micro-programmed Java Core.
 \emph{Proceedings of the 20th ACM Symposium on Applied Computing, Embedded Systems track} 879--884, March, 2005.


\subsubsection*{2004}

\item {\bf Martin Schoeberl }.
 A Time Predictable Instruction Cache for a Java Processor.
 \emph{On the Move to Meaningful Internet Systems 2004: Workshop on {J}ava Technologies for Real-Time and Embedded Systems (JTRES 2004)} 371--382, October, 2004.

\item {\bf Martin Schoeberl }.
 Design Rationale of a Processor Architecture for Predictable Real-Time Execution of Java Programs.
 \emph{Proceedings of the 10th International Conference on Real-Time and Embedded Computing Systems and Applications (RTCSA 2004)} August, 2004.

\item {\bf Martin Schoeberl }.
 Java Technology in an FPGA.
 \emph{Proceedings of the International Conference on Field-Programmable Logic and its Applications (FPL 2004)} 917--921, August, 2004.

\item {\bf Martin Schoeberl }.
 Real-Time Scheduling on a Java Processor.
 \emph{Proceedings of the 10th International Conference on Real-Time and Embedded Computing Systems and Applications (RTCSA 2004)} August, 2004.

\item {\bf Martin Schoeberl }.
 Restrictions of Java for Embedded Real-Time Systems.
 \emph{Proceedings of the 7th IEEE International Symposium on Object-Oriented Real-Time Distributed Computing (ISORC 2004)} 93--100, May, 2004.


\subsubsection*{2003}

\item {\bf Martin Schoeberl }.
 JOP: A Java Optimized Processor.
 \emph{On the Move to Meaningful Internet Systems 2003: Workshop on {J}ava Technologies for Real-Time and Embedded Systems (JTRES 2003)} 346--359, November, 2003.

\item {\bf Martin Schoeberl }.
 Design Decisions for a Java Processor.
 \emph{Tagungsband Austrochip 2003} 115--118, October, 2003.

\item {\bf Martin Schoeberl }.
 Using a Java Optimized Processor in a Real World Application.
 \emph{Proceedings of the First Workshop on Intelligent Solutions in Embedded Systems (WISES 2003)} 165--176, June, 2003.


\end{enumerate}


  
\subsection*{Technical Reports}

\begin{enumerate}

\item Jack Whitham and Martin Schoeberl.
 The limits of TDMA based memory access scheduling.
 Technical Report YCS-2011-470, University of York, 2011.

\item Martin Schoeberl, Benedikt Huber, Walter Binder, Wolfgang Puffitsch, and Alex
  Villazon.
 Object cache evaluation.
 Technical report, Technical University of Denmark, 2010.

\item Martin Schoeberl, Hiren D. Patel, and Edward A. Lee. Fun
    with a deadline instruction. Technical Report
UCB/EECS-2009-149, EECS Department, University of
  California, Berkeley, Oct 2009.
 

\end{enumerate}


\end{document}
