%\documentclass[a4paper,twocolumn]{article}
%\documentclass[10pt, conference, compsocconf]{IEEEtran}
\documentclass[a4paper, conference]{IEEEtran}


\usepackage{pslatex} % -- times instead of computer modern, especially for the plain article class
\usepackage[colorlinks=false,bookmarks=false]{hyperref}
\usepackage{booktabs}
\usepackage{graphicx}
\usepackage{xcolor}
\usepackage{multirow}
%\usepackage{flushend} % even out the last page, but use only at the end when there is a bibliography

\newcommand{\code}[1]{{\small{\texttt{#1}}}}

% fatter TT font
\renewcommand*\ttdefault{txtt}
% another TT, suggested by Alex
% \usepackage{inconsolata}
% \usepackage[T1]{fontenc} % needed as well?

\usepackage{listings}

%\newcommand{\todo}[1]{{\emph{TODO: #1}}}
\newcommand{\todo}[1]{{\color{olive} TODO: #1}}
\newcommand{\martin}[1]{{\color{blue} Martin: #1}}
\newcommand{\simon}[1]{{\color{green} Simon: #1}}
\newcommand{\abcdef}[1]{{\color{red} Author2: #1}}
\newcommand{\rewrite}[1]{{\color{red} rewrite: #1}}
\newcommand{\ducky}[1]{{\color{orange} Richard: #1}}
\newcommand{\kasper}[1]{{\color{purple} Kasper: #1}}

% uncomment following for final submission
%\renewcommand{\todo}[1]{}
%\renewcommand{\martin}[1]{}
%\renewcommand{\author2}[1]{}


%%Uncomment the following when you want to add copyright notice and not use any space	 (IEEE only)
%\usepackage[absolute]{textpos}
%% Set unit to be pagewidth and height, and increase inner margin of box
%\setlength{\TPHorizModule}{\paperwidth}\setlength{\TPVertModule}{\paperheight}
%\TPMargin{5pt}
%% Define \copyrightstatement command for easier use
%\newcommand{\copyrightstatement}{
%	\begin{textblock}{0.85}(0.072,0.93)    % Tweak here: {box width}(leftposition, rightposition)
%		\noindent
%		\normalsize
%		???-?-?-???-?/??/\$31.00~\copyright20?? IEEE % Put here your copyright
%	\end{textblock}
%}

\begin{document}


\title{Towards Verification of Digital Circuits with\\
SystemVerilog/UVM and Chisel/Scala}

\author{Martin, Simon, Kasper, Richard, and Jan}

% Most conferences have their own commands for author headings.

%\author{\IEEEauthorblockN{Edgar Lakis, Martin Schoeberl}\\
%\IEEEauthorblockA{Department of Applied Mathematics and Computer Science\\
%Technical University of Denmark\\
%Email: \texttt{edgar.lakis@gmail.com}, \texttt{masca@imm.dtu.dk}}
%}


\maketitle \thispagestyle{empty}

\begin{abstract}
Production of a chip is very expensive. Therefore, it is important to get the design right
at the first tape-out. Throughout testing and verification of the design is mandatory.
Until a few years, the two main design languages Verilog and VHDL dominated the
design and testing of digital circuits. However, both languages are decades behind
modern languages for software development.

Resent advances with SystemVerilog and Chisel bring object-oriented programming
into the digital design and verification process. SystemVerilog, as an extension of Verilog,
adds object-oriented concepts for the non-synthesizable verification code.
Chisel is a language, embedded in Scala, to describe and verify digital circuits.
Scala/Chisel brings object-oriented and functional programming into the world of
digital design.

\ducky{Chisel is a synthesis language, without dedicated test constructs - do one thing well, instead of the jack-of-both-trades Verilog/SystemVerilog}
\end{abstract}

\begin{IEEEkeywords}
digital design, verification, object-oriented programming
\end{IEEEkeywords}


\section{Introduction}
\label{sec:intro}

\todo{A brief introduction what the paper is about. It shall include briefly the
main contributions and findings. The contributions can be bullet listed.}

This paper... \todo{purpose statement, latest in 4th paragraph}

\todo{Test the bib with a reference that gives background on time-predictable
computer architecture~\cite{tpca:jes}.}

A paper is cited \cite{paper:example}.

The contributions of this paper are: (1) ... (2) ...

This paper is organized in N sections: The following section presents related work.
Section~\ref{sec:background} provides background on ...
Section X and Y 
Section~\ref{sec:eval} evaluates...
Section~\ref{sec:conclusion} concludes.

\section{Related Work}
\label{sec:related}

\todo{Show that you know the field. All related work shall be put
into context or contrast to our current work.}

\martin{Does BluseSpec Verilog has something to offer?}
\ducky{BSV is a higher-level design abstraction around guarded atomic actions - so relevant in terms of raising the level of design for digital logic, but a different approach than Chisel. My understanding (which may be outdated / wrong) is that BSV is propreitary, which is one reason it does not have significant traction. Earlier versions were heavily Haskell-based, which also does not help with perceived usability, supposedly that is why they're now branded "Bluespec SystemVerilog".}


\section{The Classic Hardware Description Languages}

\section{SystemVerilog}

\todo{adds some VHDL features to Verilog. Tries to please VHDL developers.}
\kasper{Should we focus on SystemVerilog as an HDL or as an HVL? And should we add a third heading describing the setup of a UVM testbench?} 

SystemVerilog is IEEE standard, proprietary tools.

Chisel is research stuff and open-source

\kasper{
SystemVerilog is an extension of Verilog, both of which are standardized by the IEEE. Verilog was originally standardized as IEEE 1364, and SystemVerilog is a superset of Verilog, standardized in IEEE 1800. Since 2009, they have both been standardized under IEEE 1800. 

Verilog was primarily an HDL with non-synthesizable constructs for testbenches, SystemVerilog can be seen as a Hardware Description and Verification Language (HDVL). This makes it a tool that both design and verification engineers can benefit from, and reduces the number of tools that a company must support. 

Some of the major benefits that SystemVerilog introduce include: 
\begin{itemize}
    \item Enumerated types
    \item User defined types using \texttt{typedef}
    \item Object-oriented programming for more efficient testbench design
    \item Interfaces that allow the designer to bundle related signals
\end{itemize}
Further benefits of SystemVerilog over Verilog are listed in \cite{Sutherland2010}.

For verification engineers, the fact that SystemVerilog supports Object-oriented programming (OOP), makes it easier to design reusable and scalable testbenches. By designing generic components that can be extended to provide application-specific functionality, less re-use is required.

SystemVerilog also introduces built-in support for Constrained Random Verification (CRV). In CRV, inputs to the DUT are randomized according to a set of constraints such that transactions for a given protocol can easily be generated. This saves designers the hassle of writing directed tests (is this relevant? I hope no companies are using only directed tests).

SystemVerilog also supports functional verification, allowing the verification engineer to test whether edge and corner cases have been thoroughly tested. 
}

\section{Chisel}

Chisel~\cite{chisel:dac2012} 

\section{Combining the Approaches}

Although SystemVerilog/UVM and Chisel/Scala look like competing initiatives
to solve the verification problem, it should be possible to combine those two methods.
By using this combination we might be able to enjoy the benefits of both worlds:
the industrial proven SystemVerilog/UVM tool chain with the available IPs with
the research and more software oriented approach of Chisel/Scala. 

\todo{Be nice to all and show how we can combine different approach to
get the best of both worlds}

\ducky{The main nontechnical benefits of UVM are a large library and userbase, and this inertia is probably going to hamper adoption of a new tool, especially if that new tool is more incremental than revolutionary. To my understanding, the main technical benefits of UVM are re-use and separation of responsibilities, and is inspired by an old-ish (and fairly verbose) form of OO}

\subsection{Using UVM with Chisel}

The Chisel toolchain translates Chisel code into plain Verilog for simulation and
synthesis. Therefore, we can use a UVM based test bench to test Chisel generated code.
An important issue is that the modules and port names in the generated Verilog
code are reasonable and do not change when changing the Chisel design.

\subsection{UVM with VHDL}
\kasper{Shouldn't this actually be UVM with (System)Verilog? Or are we focusing specifically on VHDL because UVM isn't made for VHDL, and thus we're looking into what's possible?}

\subsection{Chisel with Verilog}

\subsection{Chisel with VHDL}

\todo{Simon, this is the place for you}
Chisel has support for black boxes, which allows to use Verilog code within the Chisel design. However, it does not fully support VHDL. It can support VHDL using VCS, but there is no open source solution available for VHDL. This is due to the use of Treadle and Verilator \simon{Ref} for open source simulation. Treadle only supports Chisel code while Verilator is run on the generated Verilog. Therefore, Verilator can also simulate black boxes written in Verilog. However, VHDL is now a concern to companies that has a lot of source code written in VHDL, which needs to be compatible with any systems written in Chisel. As many simulation and synthesis tools supports mixed-language implementations, this is not an issue for the implementation part. But for open source testing, it proves to be a problem.

A solution for this is based on the principle of using synthesis tools to analyze the VHDL RTL code and synthesize a gate-level netlist for this. These synthesis tools can often output a Verilog-based netlist. A free solution for this would be the Yosys \simon{ref} synthesis suite, which is an open-source digital hardware synthesis suite for Verilog code. It has support for VHDL by using Verific as a front-end, which requires a license. For an alternative solution to this, GHDL \simon{ref} would be used. GHDL is an open source simulator for VHDL which also funcions as a plugin for Yosys. This allows yosys to analyze VHDL files using GHDL and then synthesize a circuit. The gate-level netlist can then be saved as a Verilog file, which can be used for the simulation system. GHDL has full support for IEEE 1076 VHDL 1987, 1993, 2002 and a subset of 2008.\\


Testing
\simon{Insert small image that depicts the workflow for this.}


\subsection{Chisel with UVM}

\todo{reusing Chisel tests (ScalaTest) within the UVM framework
with the generated Verilog code}
\ducky{UVM is largely a framework and library for testing blocks ("IP"), I don't think transpiling ChiselTest to UVM is feasible. Best I would practically hope for is interop between both, eg ChiselTest hooks to call UVM functions in a Verilog simulation environment, to enable continued use of legacy test code, Not sure the other way around is worth doing (calling ChiselTest functions from within a UVM environment)}

\kasper{ I agree with Richard on his above point. It doesn't seem feasible to use Chisel features inside of UVM. Instead, UVM might be a useful tool to verify that hardware described in Chisel lives up to its specs. 
It might be useful to focus on the fact that companies could transition their design teams into using Chisel and generating Verilog, while the verification teams wouldn't have to do anything different (assuming they already use UVM). I could definitely see this being feasible. }

\section{Co-Simulation}

\todo{with C/C++/Java/Scala/Phyton, ....}
\ducky{Should be tons of this already. CoCoTb is a Python cosimulation environment. Rocket-chip uses C++ cosimulation in their RTL tests.}

\section{Evaluation}

Although this is a work-in-progress report, we have started with an evaluation.
We implemented a simple ALU in SystemVerilog, classic Verilog, VHDL, and Chisel.
We wrote a testbench in SystemVerilog and in Scala. As execution platform we
used Synopsys VCS, ModelSim, Treadle, and Verilator.
\ducky{Likely criticism from peer review is that the example is too simple.}
\ducky{I'd also like to see some kind of human evaluation - though I'm not sure what kind of methodology is accepted in the digital design community. Could maybe drop a HCI paper cite to hint that whatever methodology used is accepted in other communities.}

The examples are available in open source from: \url{https://github.com/chisel-uvm}.


\subsection{Source Access}

\martin{I love doing papers with available source under an
open-source license. It gives credit and good karma.}


\section{Conclusion and Future Work}


This work-in-progress paper is a first sketch of the ideas to combine SystemVerilog/UVM
with Chisel/Scala for a productive design and verification of future digital circuits.
We will explore all combinations with a few small examples, provided by our industrial
partners.
From that we will bootstrap adding constraint-random verification methods to Chisel
testers and collecting coverage metrics within FIRRTL.

\subsection*{Acknowledgment}

\todo{Sometimes we received some help. Sometimes external funding.}

%This work was partially funded under the
%European Union's 7th Framework Programme
%under grant agreement no. 288008:
%Time-predictable Multi-Core Architecture for Embedded
%Systems (\mbox{T-CREST}).

%\newpage
~
\newpage


\section{Notes}

Here collect of notes and ideas in bullet list for the development and writing process:

\begin{itemize}
\item Collect related work
\item Some more notes
\end{itemize}

\subsection{Comparison}

\begin{itemize}
\item UVM is directly supported in major tools
\item UVM is based on SystemVerilog (SV)
\item OO in SV is only for test benches, not for HW description
\item SystemVerilog is three languages in one: old Verilog, new SystemVerilog for synthesize
(what constructs are supported by what tool), and SV for test benches
\item SV is a specialized language for HW design and testing
\item Chisel is two languages: Chisel for HW and Scala for tester
\item Chisel supports OO for HW description
\item Chisel is based on Scala
\item Scala is a general purpose languages
\item Scala/Chisel can use Java libraries (a LOT is available)
\item Number of Scala programmers are larger than SV programmers (how much?)
\item UVM is open source
\item SV implementation is closed source, needs commercial tools
\item UVM/SV is available on EDAPlayground
\item Chisel/Scala is open source
\item Chisel/Scala runs on the JVM, on Windows, Linux, and macOS
\item Chisel is missing the UVM methodology
\item SV (tools) supports coverage, Chisel does not
\item UVM provides functions for random constraint testing, Chisel/Scala not
\item UVM has a lot of IPs available (e.g., AXI)
\item Chisel testers need some library for sequencing and interfacing (bachelor/master thesis on parts)
\item Verdi is a GUI with knowledge of UVM
\item Verilator with C/C++ emulation is another verification option
\end{itemize}

\subsection{Research Questions}

\begin{itemize}
\item Can we use Chisel testers within UVM?
\item Can we run JVM based code in UVM?
\item Doing UVM verification on Chisel generated Verilog
\item UVM testing on Chisel interpreter (like Chisel tester using treadle, or Verilator)
\end{itemize}

\subsection{Big Research Questions}

These topics are probably good questions for a larger research project.
Part of this is probably already looked at by PhD students at UCB.

\begin{itemize}
\item Add coverage to Chisel tester
\item Provide random constraint functions
\item Provide UVM functions
\item 
\item 
\end{itemize}

\subsection{Helpful links and resources}

VHDL/SV comparison: \url{http://www.sunburst-design.com/papers/CummingsSNUG2003Boston_SystemVerilog_VHDL.pdf}

Open-source design flow, maybe for the Basys3: \url{https://symbiflow.github.io/}

Video primer on using UVM, explains what each component of the UVM test does and why things are set up the way they are
\url{https://www.youtube.com/watch?v=FkclDiK4Oco}

Website that has relatively in-depth and understandable explanations of UVM concepts and structure
\url{https://www.chipverify.com/uvm/uvm-hello-world}





\bibliographystyle{plain}
% Please do not add any references to msbib.bib.
% They get lost when I 'generate' is again (see Makefile)
\bibliography{chisel-uvm,msbib}

\end{document}

%% Adding a comment to test linking with overleaf
